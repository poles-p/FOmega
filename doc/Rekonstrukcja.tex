\documentclass[11pt,leqno]{article}

\usepackage[polish]{babel}
\usepackage[utf8]{inputenc}
\usepackage[T1]{fontenc}


\usepackage{a4wide}

\usepackage{amsfonts}
\usepackage{amsmath}
\usepackage[pdftex]{graphicx}
\usepackage{caption}
\usepackage{bm}
\usepackage{amsthm}
\usepackage{indentfirst}

 
%%%%%%%%%%%%%%%%%%
% Kropka po numerze paragrafu, podparagrafu itp. 

\makeatletter
 \renewcommand\@seccntformat[1]{\csname the#1\endcsname.\quad}
 \renewcommand\numberline[1]{#1.\hskip0.7em}
\makeatother

%%%%%%%%%%%%%%%%%%
% Kropka po numerze tablicy, rysunku i ustawienie czcionki dla etykiety. 

\captionsetup{labelfont=sc,labelsep=period}

%%%%%%%%%%%%%%%%%%
% Inna numeracja wzorów.

\renewcommand{\theequation}{\arabic{section}.\arabic{equation}}

%%%%%%%%%%%%%%%%%%

\title{{\textbf{Rekonstrukcja typów w systemie $F_{\omega}$}}\\[1ex]}
\date{Wrocław, dnia \today\ r.}
\author{Piotr Polesiuk \\ \texttt{bassists@o2.pl}}

\sloppy

\begin{document}
\thispagestyle{empty}
\maketitle

\pagebreak

\theoremstyle{plain}
\newtheorem{twierdzenie}{Twierdzenie}
\newtheorem{lemat}{Lemat}
\newtheorem{fakt}{Fakt}

\theoremstyle{definition}
\newtheorem{definicja}{Definicja}

%%%%%%%%%%%%%%%%%%%%%%%%%%%%%%%%%%%%%%%%%%%%%%%%%%%%%%%%%%%%%%%%%%%%%%%%%%%%%%%%
%%%%%%%%%%%%%%%%%%%%%%%%%%%%%%%%%%%%%%%%%%%%%%%%%%%%%%%%%%%%%%%%%%%%%%%%%%%%%%%%
\section{Kilka słów o składni.}
%%%%%%%%%%%%%%%%%%%%%%%%%%%%%%%%%%%%%%%%%%%%%%%%%%%%%%%%%%%%%%%%%%%%%%%%%%%%%%%%
%%%%%%%%%%%%%%%%%%%%%%%%%%%%%%%%%%%%%%%%%%%%%%%%%%%%%%%%%%%%%%%%%%%%%%%%%%%%%%%%
\setcounter{equation}{0}

Rozważmy język o składni:

\begin{tabular}{| l c r |}
  \hline
  t ::= &  &  \\
   & $x$ & \textit{zmienne}  \\
   & $\lambda x:T.t$ & \textit{abstrakcja} \\
   & $t\;t$ & \textit{aplikacja} \\
   & $\Delta X::K.t$ & \textit{abstrakcja typowa} \\
   & $t[T]$ & \textit{aplikacja typowa} \\
   & $\textrm{let} \ x = t \ \textrm{in} \ t$ & \textit{zmienna lokalna} \\
   & $\textrm{tlet} \ X = T \ \textrm{in} \ t$ & \textit{typ lokalny} \\
   & & \\
  T ::= &  &  \\
   & $X$ & \textit{zmienna typowa (kwantyfikowana abstrakcją)} \\
   & $T \rightarrow T$ & \textit{typ funkcji} \\
   & $\forall X::K.T$ & \textit{typ uniwersalny} \\
   & $\Lambda X::K.T$ & \textit{konstruktor abstrakcji typowej} \\
   & $T\;T$ & \textit{aplikacja konstruktora typowego} \\
   & $\bar{X}$ & \textit{zmienna typowa (kwantyfikowana schematem)} \\
   & & \\
  K ::= & & \\
   & $*$ & \textit{rodzaj wszystkich typów} \\
   & $K \Rightarrow K$ & \textit{rodzaj funkcji typowej} \\
   & $\widehat{X}$ & \textit{zmienna rodzajowa (kwantyfikowana schematem)} \\
   & & \\
  \hline
\end{tabular} \\

W dodatku zakładamy, że wszystkie zmienne związane kwantyfikowane abstrakcją są unikatowe (tzn. zmienna $X$ może
występować za conajwyżej jednym kwantyfikatorem).

Ważnym założeniem jest to, że $\bar{X}::*$, co nam ułatwi sprawę przy unifikacji, a~zarazem zbytnio nie ograniczy języka.
  
Jeżeli chcemy dodać ML-polimorfizm, będą nam portrzebne schematy typów i schematy rodzajów:

\begin{tabular}{| l c r |}
  \hline
  $\left<T\right>$ ::= &  & \textit{schemat typu} \\
   & $T$ & \textit{typ schematu} \\
   & $\Omega \bar{X}.\left<T\right>$ & \textit{kwantyfikacja zmiennej typowej} \\
   & $\Omega \widehat{X}.\left<T\right>$ & \textit{kwantyfikacja zmiennej rodzajowej} \\
   & & \\
  $\left<K\right>$ ::= &  & \textit{schemat rodzaju} \\
   & $K$ & \textit{rodzaj schematu}\\
   & $\Omega \widehat{X}.\left<K\right>$ & \textit{kwantyfikacja zmiennej rodzajowej} \\
   & & \\
  \hline
\end{tabular} \\

Będziemy używać następującego cukru syntaktycznego:
\[
\Omega \bar{X}_1 \dots \bar{X}_n \widehat{X}_1 \dots \widehat{X}_m.T \equiv 
\Omega \bar{X}_1. \dots \Omega \bar{X}_n. \Omega \widehat{X}_1. \dots \Omega \widehat{X}_m.T
\]
I analogicznie dla rodzajów.

Teraz sam kontekst typowania ma postać:

\begin{tabular}{| l c r |}
  \hline
  $\Gamma$ ::= &  & \\
   & $\emptyset$ & \\
   & $\Gamma, x:\left<T\right>$ & \\
   & $\Gamma, X::\left<K\right>$ & \\
   & & \\
  \hline
\end{tabular} \\

Jeżeli chcemy spojrzeć na ten język, jak na język programowania z~rekonstrukcją typów, można
dopuścić nieanotowane kwantyfikatory, które parser będzie zamieniać na anotowane świeżą zmienną schematową 
(typową albo rodzajową w~zależności od kwantyfikatora). Z~drugiej strony nie ma potrzeby pozwalać programiście
jawnie korzystać ze zmiennych schematowych.

\section{Zbiory zmiennych i~podstawienia}

W~dalszej części będziemy używać różnych zbiorów zmiennych występujących w~konstrukcjach w~naszym języku, więc
to jest dobre miejsce by te zbiory i~operacje zdefiniować.

\subsection{Wszystkie zmienne schematowe}

\begin{definicja}
\emph{Zbiorem schematowych zmiennych typowych typu/schematu $T$} nazwiemy zbiór $\bar{Var}(T)$ zdefiniowany następująco
\begin{align*}
\bar{Var}(X) & = \emptyset \\
\bar{Var}(T_1 \to T_2) & = \bar{Var}(T_1) \cup \bar{Var}(T_2) \\
\bar{Var}(\forall X :: K.T) & = \bar{Var}(T) \\
\bar{Var}(\Lambda X :: K.T) & = \bar{Var}(T) \\
\bar{Var}(T_1 \; T_2) & = \bar{Var}(T_1) \cup \bar{Var}(T_2) \\
\bar{Var}(\bar{X}) & = \{ \bar{X} \}
\end{align*}
\begin{align*}
\bar{Var}(\Omega \bar{X}.\left<T\right>) & = \{ \bar{X} \} \cup \bar{Var}(\left<T\right>) \\
\bar{Var}(\Omega \widehat{X}.\left<T\right>) & = \bar{Var}(\left<T\right>)
\end{align*}
\end{definicja}

\begin{definicja}
Anologicznie definijuemy \emph{zbiór zmiennych rodzajowych rodzaju/typu/schematu $T$}.
\begin{align*}
\widehat{Var}(*) & = \emptyset \\
\widehat{Var}(K_1 \Rightarrow K_2) & = \widehat{Var}(K_1) \cup \widehat{Var}(K_2) \\
\widehat{Var}(\widehat{X}) & = \{ \widehat{X} \}
\end{align*}
\begin{align*}
\widehat{Var}(X) & = \emptyset \\
\widehat{Var}(T_1 \to T_2) & = \widehat{Var}(T_1) \cup \widehat{Var}(T_2) \\
\widehat{Var}(\forall X :: K.T) & = \widehat{Var}(K) \cup \widehat{Var}(T) \\
\widehat{Var}(\Lambda X :: K.T) & = \widehat{Var}(K) \cup \widehat{Var}(T) \\
\widehat{Var}(T_1 \; T_2) & = \widehat{Var}(T_1) \cup \widehat{Var}(T_2) \\
\widehat{Var}(\widehat{X}) & = \emptyset
\end{align*}
\begin{align*}
\widehat{Var}(\Omega \bar{X}.\left<T\right>) & = \widehat{Var}(\left<T\right>) \\
\widehat{Var}(\Omega \widehat{X}.\left<T\right>) & = \{ \widehat{X} \} \cup \widehat{Var}(\left<T\right>)
\end{align*}
\end{definicja}

\subsection{Zmienne typowe wolne i~związane}

\begin{definicja}
\emph{Zbiór zmiennych typowych wolnych termu/typu $T$} nazwiemy zbiór $FTV(T)$ zdefiniowany następująco
\begin{align*}
FTV(X) & = X \\
FTV(T_1 \to T_2) & = FTV(T_1) \cup FTV(T_2) \\
FTV(\forall X :: K . T) & = FTV(T) \setminus \{ X \} \\
FTV(\Lambda X :: K . T) & = FTV(T) \setminus \{ X \} \\
FTV(T_1 \; T_2) & = FTV(T_1) \cup FTV(T_2) \\
FTV(\bar{X}) & = \emptyset \\
\end{align*}
\begin{align*}
FTV(x) & = \emptyset \\
FTV(\lambda x : T.t) & = FTV(T) \cup FTV(t) \\
FTV(t_1 \; t_2) & = FTV(t_1) \cup FTV(t_2) \\
FTV(\Delta X :: K.t) & = FTV(t) \setminus \{ X \} \\
FTV(t[T]) & = FTV(t) \cup FTV(T) \\
FTV(\textrm{let} \ x = t_1 \ \textrm{in} \ t_2) & = FTV(t_1) \cup FTV(t_2) \\
FTV(\textrm{tlet} \ X = T \ \textrm{in} \ t) & = FTV(T) \cup (FTV(t) \setminus \{ X \}) \\
\end{align*}
\end{definicja}

\begin{definicja}
\emph{Zbiór zmiennych typowych związanych termu/typu $T$} nazwiemy zbiór $BTV(T)$ zdefiniowany następująco
\begin{align*}
BTV(X) & = \emptyset \\
BTV(T_1 \to T_2) & = BTV(T_1) \cup BTV(T_2) \\
BTV(\forall X :: K . T) & = BTV(T) \cup \{ X \} \\
BTV(\Lambda X :: K . T) & = BTV(T) \cup \{ X \} \\
BTV(T_1 \; T_2) & = BTV(T_1) \cup BTV(T_2) \\
BTV(\bar{X}) & = \emptyset \\
\end{align*}
\begin{align*}
BTV(x) & = \emptyset \\
BTV(\lambda x : T.t) & = BTV(T) \cup BTV(t) \\
BTV(t_1 \; t_2) & = BTV(t_1) \cup BTV(t_2) \\
BTV(\Delta X :: K.t) & = BTV(t) \cup \{ X \} \\
BTV(t[T]) & = BTV(t) \cup BTV(T) \\
BTV(\textrm{let} \ x = t_1 \ \textrm{in} \ t_2) & = BTV(t_1) \cup BTV(t_2) \\
BTV(\textrm{tlet} \ X = T \ \textrm{in} \ t) & = BTV(T) \cup BTV(t) \cup \{ X \} \\
\end{align*}
\end{definicja}

\subsection{Podstawienia}

Założenie o~unikalności zwmiennych związanych wiąże się z~niewielką modyfkacją podstawienia.
\[
\{X := T\} Y = 
	\begin{cases}
		\alpha T & X=Y, \alpha \ \textrm{jest przemianowaniem zmiennych} \\
		& \textrm{związanych w~$T$ na świeże} \\
		Y & X \neq Y
	\end{cases} \\
\]
Reszta definicji pozostaje bez zmian.

Przy rekonstrukcji typów będziemy dodatkowo używać podstawienia, które dodatkowo wstawia świeże zmienne rodzajowe.
Zdefiniujemy je podobnie:
\[
\{X(\{\widehat{X}_1, \dots, \widehat{X}_n\}) := T\} Y = 
	\begin{cases}
		\alpha \sigma T & X=Y, \alpha \ \textrm{jest przemianowaniem zmiennych} \\
		& \textrm{związanych w~$T$ na świeże}, \\
		& \sigma = [\widehat{X}_1 := fresh, \dots, \widehat{X}_n := fresh] \\
		Y & X \neq Y
	\end{cases} \\
\]
\[
\ldots
\]
$\sigma$ jest podstawieniem schematowym, definicja jego działania pojawi się później.

Purystów matematycznych może przerażać pojawiająca się w~definicji ,,świeża zmienna'', oznaczana
jako $fresh$. Jest to zmienna która dotychczas nigdzie się nie pojawiła, i~nie będzie wprowadzona więcej niż raz.
Można ją formalnie zdefiniować pamiętając cały czas zbiór użytych zmiennych i~przekazując go jako dodatkowy argument do prawie
wszystkich operacji, ale spowodowało by to znaczne pogorszenie czytelności tego dokumentu, dlatego tak nie robimy.
Zainteresowany Czytelnik może samodzielnie poprawić przytoczone tu definicje i~twierdzenia formalnie definiując świeżą zmienną.

Dla przytoczonych tu definicji później zajdzie potrzeba rozszerzenia ich na konteksty i~podstawienia.
Rozszerzenia te są na tyle oczywiste i~naturalne, że nie ma potrzeby ich przytaczania.

\section{$\beta$-unifikacja}

Podczas rekonstukcji typów pojawiają się równania więzów które należy rozwiązać. Samo rozwiązanie
sprowadza się do unifikacji pewnych termów, ale tutaj, ze względu na możliwość występowania funkcji
typowych, termy równoważne nie muszą być równe, więc sama unifikacja powinna sprowadzać termy do
$\beta$-równych sobie.

Taka unifikacja niesie ze sobą wiele problemów. Po pierwsze, będziemy chcieli używać podstawienia (za zmienne schematowe),
będącego unifikatorem, na innych termach. To zaś grozi uzewnętrznieniem zmiennych związanych. Drugi
problem to taki, że samo podstawienie nie musi zrównywać termów, które da się zrównać, ale różnią się
nazwami zmiennych związanych.

Z pierwszym problemem poradzimy sobie, traktując podstawienia jako funkcję częściowe, tzn. samo podstawienie
oprócz przyporzątkowań postaci $[\bar{X} := T]$, może również zawierać podstawienia postaci $[\bar{X} := fail]$.
Oraz dodając jeszcze operację anulowania podstawienia:
\[
[\bar{X} := T] \setminus X = 
\begin{cases}
[\bar{X} := fail] & X \in    FTV(T) \\
[\bar{X} := T]    & X \notin FTV(T)
\end{cases}
\]

Z drugim problemem poradzimy sobie traktując unifikator jako parę zawierającą podstawienie, które dobrze działa
na zewnątrz, oraz term będący wynikiem unifikacji.

\begin{definicja}
\emph{Podstawieniem schematowym} nazwiemy skończony zbiór par postaci $(\bar{X},T)$, $(\bar{X},fail)$ oraz $(\widehat{X},K)$, takich że
$\bar{X} \notin \bar{Var}(T)$ oraz $\widehat{X} \notin \widehat{Var}(K)$ (po podstawieniu powinny zniknąć wszystkie zmienne, za 
które coś podstawiliśmy). 
\emph{Podstawieniem schematowym pustym} nazwiemy podstawienie schematowe będące zbiorem pustym i~będziemy oznaczać przez $[]$.
Podstawienia schematowe, będziemy reprezentować jako listy postaci
\[
[\bar{X}_1 := T_1, \dots, \bar{X}_k := T_k, \bar{X}_{k+1} := fail, \dots , \bar{X}_n := fail, \widehat{X}_1 := K_1, \dots, \widehat{X}_m := K_m]
\]
\emph{Dziedziną} podstawienia schematowego $\sigma$ nazwiemy zbiór takich zmiennych typowych $\bar{X}$ i~rodzajowych $\widehat{X}$, że
$(\bar{X}, T) \in \sigma$ lub $(\bar{X}, fail) \in \sigma$ oraz $(\widehat{X}, K) \in \sigma$. Dziedzinę podstawienia schematowego $\sigma$
będziemy oznaczać przez $Dom(\sigma)$.
\end{definicja}

\begin{definicja}
Podstawienie schematowe może działać na rodzajach, typach, termach i~kontekstach. Jeżeli w~podstawieniu występuje
para $\widehat{X} := K$, to za wszystkie wystąpienia zmiennej~$\widehat{X}$ zostanie podstawione $K$. Analogicznie
się dzieje w~przypadku wystąpienia pary $\bar{X} := T$. Jeżeli w~podstawieniu schematowym wystąpi para $\bar{X} := fail$,
a~w~typie/termie/kontekście do którego podstawienie aplikujemy występuje zmienna $\bar{X}$, to podstawienie jest niemożliwe.

Niech $\sigma$ będzie podstawieniem schematowym. Działanie podstawienia formalnie definiujemy jako:
\begin{align*}
\sigma * & = * \\
\sigma (K_1 \Rightarrow K_2) & = \sigma K_1 \Rightarrow \sigma K_2 \\
\sigma \widehat{X} & = 
	\begin{cases}
		K & (\widehat{X},K) \in \sigma \\
		\widehat{X} & \textrm{wpp}
	\end{cases} \\
\end{align*}
\begin{align*}
\sigma X & = X \\
\sigma(T_1 \to T_2) & = \sigma T_1 \to \sigma T_2 \\
\sigma(\forall X :: K.T) & = \forall X :: \sigma K. \sigma T \\
\sigma(\Lambda X :: K.T) & = \Lambda X :: \sigma K. \sigma T \\
\sigma(T_1 \; T_2) & = (\sigma T_1) \; (\sigma T_2) \\
\sigma \bar{X} & = 
	\begin{cases}
		T & (\bar{X},T) \in \sigma \\
		fail & (\bar{X},fail) \in \sigma \\
		\bar{X} & \textrm{wpp}
	\end{cases} \\
\end{align*}
\begin{align*}
\sigma x & = x \\
\sigma (\lambda x : T.t) & = \lambda x : \sigma T. \sigma t \\
\sigma (t_1 \; t_2) & = (\sigma t_1) \; (\sigma t_2) \\
\sigma (\Delta X :: K. t) & = \Delta X :: \sigma X . \sigma t \\
\sigma (t[T]) & = (\sigma t)[\sigma T] \\
\sigma (\textrm{let} \ x = t_1 \ \textrm{in} \ t_2) & = \textrm{let} \ x = \sigma t_1 \ \textrm{in} \ \sigma t_2 \\
\sigma (\textrm{tlet} \ x = T \ \textrm{in} \ t) & = \textrm{tlet} \ x = \sigma T \ \textrm{in} \ \sigma t \\
\end{align*}
\begin{align*}
\sigma (\Omega\bar{X}.\left<T\right>) & =
	\begin{cases}
		\Omega\bar{X}.\sigma\left<T\right> & \bar{X} \notin Dom(\sigma) \\
		\Omega\bar{Y}.\sigma[\bar{X}:=\bar{Y}]\left<T\right> & \bar{X} \in Dom(\sigma) \land \bar{Y} \notin Dom(\sigma) \land \bar{Y} \notin \bar{Var}(\left<T\right>)
	\end{cases} \\
\sigma (\Omega\widehat{X}.\left<T\right>) & =
	\begin{cases}
		\Omega\widehat{X}.\sigma\left<T\right> & \widehat{X} \notin Dom(\sigma) \\
		\Omega\widehat{Y}.\sigma[\widehat{X}:=\widehat{Y}]\left<T\right> & \widehat{X} \in Dom(\sigma) \land \widehat{Y} \notin Dom(\sigma) \land \widehat{Y} \notin \widehat{Var}(\left<T\right>)
	\end{cases}
\end{align*}
\begin{align*}
\sigma \emptyset & = \emptyset \\
\sigma (\Gamma, x : \left<T\right>) & = \sigma \Gamma, x : \sigma \left<T\right> \\
\sigma (\Gamma, X :: \left<K\right>) & = \sigma \Gamma, X :: \sigma \left<K\right> \\
\end{align*}
\end{definicja}

\begin{definicja}
Powiemy, że podstawienie schematowe $\rho$ jest \emph{złożeniem} podstawień schematowych $\sigma$ i~$\theta$, 
które będziemy oznaczać $\sigma\circ\theta$ jeżeli
dla każdego rodzaju $K$, typu $T$, termu $t$ i~kontekstu $\Gamma$ zachodzi:
\begin{align*}
\rho K & = \sigma(\theta K) \\
\rho T & = \sigma(\theta T) \\
\rho t & = \sigma(\theta t) \\
\rho \Gamma & = \sigma(\theta \Gamma)
\end{align*}
\end{definicja}

\begin{lemat}
Niech $\sigma$ i~$\theta$ będą podstawieniami schematowymi. Wówczas zachodzi
\begin{align*}
\sigma \circ \theta = & \{ (\widehat{X}, \sigma K) | (\widehat{X}, K) \in \theta \} \\
\cup & \{ (\widehat{X}, K) \in \sigma | \widehat{X} \notin Dom(\theta) \} \\
\cup & \{ (\bar{X}, \sigma T) | (\bar{X}, T) \in \theta \land \sigma T \neq fail \} \\
\cup & \{ (\bar{X}, fail) | \exists (\bar{X}, T) \in \theta . \sigma T = fail \} \\
\cup & \{ (\bar{X}, T) \in \sigma | \bar{X} \notin Dom(\theta) \} \\
\cup & \{ (\bar{X}, fail) \in \sigma | \bar{X} \notin Dom(\theta) \}
\end{align*}
Pozwala nam to algorytmicznie wyliczać złożenia podstawień schematowych.
\end{lemat}
\begin{proof}
\end{proof}

\begin{definicja}
\emph{Anulowaniem zbioru $A$ zmiennych typowych kwantyfikowanych abstrakcją z~podstawienia schematowego $\sigma$} nazwiemy operację
zdefiniowaną następująco:
\begin{align*}
\sigma \setminus A = & \{ (\widehat{X}, K) \in \sigma \} \cup \{ (\bar{X}, fail) \in \sigma \} \\
\cup & \{ (\bar{X}, T) \in \sigma | FTV(T) \cap A = \emptyset \} \\
\cup & \{ (\bar{X}, fail) | \exists (\bar{X}, T) \in \sigma . FTV(T) \cap A \neq \emptyset \}
\end{align*}
\emph{Anulowanie zmiennej $X$ z~podstawienia schematowego $\sigma$} oznaczamy przez $\sigma \setminus X$ i~definiujemy jako $\sigma \setminus \{X\}$.
\end{definicja}

\begin{definicja}
\emph{$\beta$-unifikatorem} dla konstruktorów typów $T_1$ i $T_2$ nazwiemy taką parę $(\sigma, S)$, że istnieją
podstawienia schematowe $\theta_1$ i $\theta_2$ takie, że:
\[
\theta_1 T_1 =_\beta S =_\beta \theta_2 T_2
\]
oraz
\[
\theta_1 \setminus BTV(T_1) = \sigma = \theta_2 \setminus BTV(T_2)
\]
\end{definicja}

\begin{definicja}
Powiemy, że $\beta$-unifikator $(\sigma, S)$ jest \emph{ogólniejszy} od $\beta$-unifikatora
$(\theta, T)$, jeżeli istnieje takie podstawienie schematowe $\rho$ i~przemianowanie $\alpha$, że
\[
\rho\alpha S =_\beta T \quad \textrm{oraz} \quad \rho\alpha\sigma = \theta
\]
\end{definicja}

\begin{definicja}
\emph{Najogólniejszym $\beta$-unifikatorem} dla konstruktorów typów $T_1$ i $T_2$ nazwiemy taki $\beta$-unifikator,
który jest ogólniejszy od wszystkich innych $\beta$-unifikatorów tychże konstruktorów typów.
\end{definicja}

\begin{lemat}
Niech $A$ i $B$ będą podzbiorami zbioru zmiennych typowych kwnatyfikowanych abstrakcją, niech $\theta$ będzie podstawieniem schematowym.
Wówczas zachodzi
\[
(\theta \setminus A) \setminus B = \theta \setminus (A \cup B).
\]
\label{lsub0}
\end{lemat}
\begin{proof}
\end{proof}

\begin{lemat}
Niech $A$ będzie podzbiorem zbioru zmiennych typowych kwnatyfikowanych abstrakcją, niech $\theta$ i~$\rho$ będą podstawieniami schematowymi, oraz
niech zachodzi $FTV(\rho) \cap A = \emptyset$. Wówczas zachodzi 
\[
(\rho \circ \theta) \setminus A = \rho \circ (\theta \setminus A)
\].
\label{lsub1}
\end{lemat}
\begin{proof}
\end{proof}

\begin{lemat}
Niech $A$ będzie podzbiorem zbioru zmiennych typowych kwnatyfikowanych abstrakcją, niech $\theta$ będzie dowolnym podstawieniem schematowym,
niech $\rho$ będzie podstawieniem schematowym takim, że $FTV(\rho) \cap A = \emptyset$ (w~szczególności zawierającym tylko zmienne rodzajowe). 
Wówczas zachodzi
\[
(\theta \circ \rho) \setminus A = (\theta \setminus A) \circ \rho
\]
\label{lsub2}
\end{lemat}
\begin{proof}
\end{proof}

\begin{lemat}
Niech $(\sigma, S)$ będzie $\beta$-unifikatorem dla konstruktorów typów $T_1$ i~$T_2$, oraz niech $\rho$ będzie 
podstawieniem schematowym takim, że \mbox{$FTV(\rho) \cap BTV(T_1) = \emptyset$} oraz \mbox{$FTV(\rho) \cap BTV(T_2) = \emptyset$}.
Wówczas $(\rho \circ \sigma, \rho S)$ również jest $\beta$-unifikatorem dla konstruktorów typów $T_1$ i~$T_2$.
\label{lunif1}
\end{lemat}
\begin{proof}
Wiemy, że istnieją takie $\theta_1$ i~$\theta_2$, że
\[
\theta_1 T_1 =_\beta S =_\beta \theta_2 T_2 \quad \textrm{oraz} \quad \theta_1 \setminus BTV(T_1) = \sigma = \theta_2 \setminus BTV(T_2)
\]
A wtedy zachodzi
\[
(\rho \circ \theta_1)T_1 = \rho(\theta_1 T_1) =_\beta \rho S =_\beta \rho(\theta_2 T_2) = (\rho \circ \theta_2)T_2
\]
oraz z lematu \ref{lsub1}
\[
(\rho \circ \theta_1) \setminus BTV(T_1) = \rho \circ (\theta_1 \setminus BTV(T_1)) = \rho \circ \sigma
\]
\[
(\rho \circ \theta_2) \setminus BTV(T_2) = \rho \circ (\theta_2 \setminus BTV(T_2)) = \rho \circ \sigma
\]
\end{proof}

\begin{lemat}
Niech $T_1$ i~$T_2$ będą konstrktorami typów, $\rho$ niech będzie podstawieniem schematowym takim, że 
$FTV(\rho) \cap BTV(T_1) = \emptyset$ oraz $FTV(\rho) \cap BTV(T_2) = \emptyset$, a $(\sigma, S)$ niech będzie 
\mbox{$\beta$-unifikatorem} konstruktorów $(\rho T_1, \rho T_2)$.
Wówczas $(\sigma \circ \rho, S)$ jest $\beta$-unifikatorem konstruktorów $(T_1, T_2)$.
\label{lunif2}
\end{lemat}
\begin{proof}
Niech $i \in \left\{1,2\right\}$.\\
Z tego, że $(\sigma, S)$ jest $\beta$-unifikatorem konstruktorów $(\rho T_1, \rho T_2)$ wiemy, że istnieje takie
$\theta_i$, że
\[
\theta_i\rho T_i =_\beta S \quad \textrm{oraz} \quad \theta_i \setminus BTV(\rho T_i) = \sigma
\]
A~wtedy
\[
(\theta_i \circ \rho) T_i =_\beta S
\]
oraz z~lematu \ref{lsub2} i~z~tego że $BTV(\rho) \cap BTV(\theta_i) = \emptyset$ mamy
\[
(\theta_i \circ \rho) \setminus BTV(T_i) = (\theta_i \circ BTV(T_i)) \circ \rho = (\theta_i \circ BTV(\rho T_i)) \circ \rho = \sigma \circ \rho
\]
Co kończy dowód.
\end{proof}

\subsection{Algorytm $\beta$-unifikacji}

Przy unifikacji wszystkie nieanotowane kwantyfikatory traktujemy jak anotowane unikatową zmienną.
Funkcja $unify$ to klasyczna unifikacja (na rodzajach).\\
\begin{tabular}{l}
$unify_\beta(T_1, T_2) \ \textrm{when} \ T_1 =_\beta T_2 \ = $ \\
$\qquad ([], T_1)$ \\
\end{tabular} \\
\begin{tabular}{l}
$unify_\beta(\bar{X}, T) = $ \\
$\qquad \textrm{if} \ \bar{X} \in \bar{Var}(T) \ \textrm{then} \ fail $ \\
$\qquad \textrm{else} \ ([\bar{X} := T], T) $ \\
\end{tabular} \\
\begin{tabular}{l}
$unify_\beta(T, \bar{X}) = $ \\
$\qquad \textrm{if} \ \bar{X} \in \bar{Var}(T) \ \textrm{then} \ fail $ \\
$\qquad \textrm{else} \ ([\bar{X} := T], T) $ \\
\end{tabular} \\
\begin{tabular}{l}
$unify_\beta(\forall X_1::K_1.T_1, \forall X_2::K_2.T_2) = $ \\
$\qquad \textrm{let} \ \sigma_K = unify(K_1, K_2) \ \textrm{in} $ \\
$\qquad \textrm{let} \ (\sigma_T, T) = unify_\beta(\sigma_K T_1, \sigma_K \{X_2 := X_1\}T_2) \ \textrm{in} $ \\
$\qquad\qquad ((\sigma_T \setminus X_1) \circ \sigma_K, \forall X_1::\sigma_T \sigma_K K_1.T) $ \\
\end{tabular} \\
\begin{tabular}{l}
$unify_\beta(\Lambda X_1::K_1.T_1, \Lambda X_2::K_2.T_2) = $ \\
$\qquad \textrm{let} \ \sigma_K = unify(K_1, K_2) \ \textrm{in} $ \\
$\qquad \textrm{let} \ (\sigma_T, T) = unify_\beta(\sigma_K T_1, \sigma_K \{X_2 := X_1\}T_2) \ \textrm{in} $ \\
$\qquad\qquad ((\sigma_T \setminus X_1) \circ \sigma_K, \Lambda X_1::\sigma_T \sigma_K K_1.T) $ \\
\end{tabular} \\
\begin{tabular}{l}
$unify_\beta(T_1 \rightarrow S_1, T_2 \rightarrow S_2) = $ \\
$\qquad \textrm{let} \ (\theta, T) = unify_\beta(T_1, T_2) \ \textrm{in} $ \\
$\qquad \textrm{let} \ (\sigma, S) = unify_\beta(\theta S_1, \theta S_2) \ \textrm{in} $ \\
$\qquad\qquad (\sigma \circ \theta, \sigma T \rightarrow S) $ \\
\end{tabular} \\
\begin{tabular}{l}
$unify_\beta(T_1, T_2) \ \textrm{when} \ T_1 \longrightarrow_\beta S_1 \ = $ \\
$\qquad unify_\beta(S_1, T_2)$ \\
\end{tabular} \\
\begin{tabular}{l}
$unify_\beta(T_1, T_2) \ \textrm{when} \ T_2 \longrightarrow_\beta S_2 \ = $ \\
$\qquad unify_\beta(T_1, S_2)$ \\
\end{tabular} \\
\begin{tabular}{l}
$unify_\beta(V_1 \; T_1, V_2 \; T_2) = $ \\
$\qquad \textrm{let} \ (\theta, V) = unify_\beta(V_1, V_2) \ \textrm{in} $ \\
$\qquad \textrm{let} \ (\sigma, T) = unify_\beta(\theta T_1, \theta T_2) \ \textrm{in} $ \\
$\qquad\qquad (\sigma \circ \theta, \sigma V \; T) $ \\
\end{tabular} \\

\subsection{Własności}

\begin{fakt}
Algorytm $unify_\beta$ dla konstruktorów typów, dla których daje się określić rodzaj, zatrzymuje się.
\end{fakt}
\begin{proof}
Trywialne. Wynika z~tego, że algorytm jest sterowany składnią, oraz z~silnej normalizowalności kostruktorów typów.
\end{proof}
\begin{twierdzenie}
Niech $T_1$ oraz $T_2$ będą konstruktorami typów, które daje się orodzajować. Wówczas jeśli
$unify_\beta(T_1, T_2) = (\sigma, S)$, to $(\sigma, S)$ jest $\beta$-unifikatorem konstruktorów typów
$T_1$ i~$T_2$.
\end{twierdzenie}
\begin{proof}
Indukcja po głebokości rekursji.

Możliwe są następujące przypadki:
\begin{itemize}
\item $T_1 =_\beta T_2$ lub $T_1 \equiv \bar{X}$ lub $T_2 \equiv \bar{X}$ \\
dla tych przypadków teza zachodzi trywialnie.
\item $T_1 \equiv \forall X_1::K_1.T'_1$ oraz $T_2 \equiv \forall X_2::K_2.T'_2$ \\
Wtedy z załorzenia indukcyjnego $(\sigma_T, T)$ jest $\beta$-unifikatorem konstruktorów typów
$\sigma_K T'_1$ oraz $\sigma_K\{X_2:=X_1\}T'_2$, gdzie $\sigma_K$ jest unifikatorem rodzajów $K_1$ i~$K_2$.
Istnieją więc takie podstawienia schematowe $\theta_1$ i~$\theta_2$, że zachodzą równości
\[
\theta_1 \sigma_K T'_1 =_\beta T =_\beta \theta_2 \sigma_K \{X_2:=X_1\} T'_2
\]
\[
\theta_1 \setminus BTV(\sigma_K T'_1) = \sigma_T = \theta_2 \setminus BTV(\sigma_K \{X_2:=X_1\} T'_2)
\]
A wtedy
\[
(\theta_1 \circ \sigma_K)\forall X_1::K_1.T'_1 = \forall X_1::\theta_1\sigma_K K_1. \theta_1\sigma_K T'_1 =_\beta
\]
\[
=_\beta \forall X_1::(\theta_1 \setminus BTV(\sigma_K T'_1))\sigma_K K_1.T = \forall X_1::\sigma_T\sigma_K K_1.T
\]
oraz
\[
(\theta_1 \circ \sigma_K) \setminus BTV(\forall X_1::K_1.T'_1) = (\theta_1 \setminus (\{X_1\} \cup BTV(\sigma_K T'_1))) \circ \sigma_K =
\]
\[
= (\sigma_T \setminus X_1) \circ \sigma_K
\]
Analogicznie pokarzemy pozostałe dwie równości dla konstruktora $T_2$. Zatem 
\mbox{$((\sigma_T \setminus X_1) \circ \sigma_K, \forall X_1::\sigma_T \sigma_K K_1.T)$} jest $\beta$-unifikatorem konstruktorów $T_1$ i~$T_2$.
\item $T_1 \equiv \Lambda X_1::K_1.T'_1$ oraz $T_2 \equiv \Lambda X_2::K_2.T'_2$ \\
Dla tego przypadku dowód przeprowadzamy analogicznie do przypadku poprzedniego.
\item $T_1 \equiv T'_1 \rightarrow S_1$ oraz $T_2 \equiv T'_2 \rightarrow S_2$ \\
Teza wynika wprost z~załorzenia indukcyjnego oraz lematów \ref{lunif1} i \ref{lunif2}.
\item $T_1 \longrightarrow_\beta S_1$ lub $T_2 \longrightarrow_\beta S_2$ \\
Wynika z~załorzenia indukcyjnego i~z~faktu, że $\beta$-redukcja zachowuje $\beta$-równość.
\item $T_1 \equiv V_1\;S_1$ oraz $T_2 \equiv V_2\;S_2$ \\
Dowód przeprowadzamy analogicznie do przypadku z typem funkcji.
\end{itemize}
\end{proof}

\begin{fakt}
Algorytm $unify_\beta$ działający na konstruktorach typów nie zawierających zmiennych schematowych
zwraca pusty unifikator jeśli konstruktory są $\beta$ równe, w~przeciwnym przypadku zawodzi (zawraca $fail$).
\end{fakt}
\begin{proof}
\end{proof}

\begin{fakt}
Jeżeli w~algorytmie pierwszy przypadek sprawdzający $\beta$-równość konstruktorów zamienimy na analogiczy 
sprawdzający tożsamość zmiennych typowych (a~przy rozszerzeniach jeszcze typów bazowych), otrzymamy algorytm
równoważny.
\end{fakt}
\begin{proof}
\end{proof}

Tu powinno pojawić się jeszcze kilka dowodów własności.

\section{Rekonstrukcja typów}

\subsection{Algorytm W}

Dla rodzajów: \\
\begin{tabular}{l}
$kindof(\Gamma \vdash X :: ?) = $ \\
$\qquad \textrm{let} \ \Omega \widehat{X}_1 \dots \widehat{X}_n.K = \Gamma(X) \ \textrm{in} $ \\
$\qquad \textrm{let} \ \widehat{Y}_1, \dots, \widehat{Y}_n = fresh \ \textrm{in} $ \\
$\qquad\qquad ([], [\widehat{X}_1 := \widehat{Y}_1, \dots ,\widehat{X}_n := \widehat{Y}_n]K) $ \\
\end{tabular} \\
\begin{tabular}{l}
$kindof(\Gamma \vdash T_1 \rightarrow T_2 :: ?) = $ \\
$\qquad \textrm{let} \ (\sigma_1, K_1) = kindof(\Gamma \vdash T_1 :: ?) \ \textrm{in} $ \\
$\qquad \textrm{let} \ (\sigma_2, K_2) = kindof(\sigma_1 \Gamma \vdash \sigma_1 T_1 :: ?) \ \textrm{in} $ \\
$\qquad \textrm{let} \ \rho_1 = unify(\sigma_2 K_1, *) \ \textrm{in} $ \\
$\qquad \textrm{let} \ \rho_2 = unify(\rho_1 K_2, *) \ \textrm{in} $ \\
$\qquad\qquad (\rho_2 \circ \rho_1 \circ \sigma_2 \circ \sigma_1, *) $ \\
\end{tabular} \\
\begin{tabular}{l}
$kindof(\Gamma \vdash \forall X::K.T :: ?) = $ \\
$\qquad \textrm{let} \ (\sigma, K_T) = kindof(\Gamma, X::K \vdash T :: ?) \ \textrm{in} $ \\
$\qquad \textrm{let} \ \rho = unify(K_T, *) \ \textrm{in} $ \\
$\qquad\qquad (\rho \circ \sigma, *) $ \\
\end{tabular} \\
\begin{tabular}{l}
$kindof(\Gamma \vdash \Lambda X::K.T :: ?) = $ \\
$\qquad \textrm{let} \ (\sigma, K_T) = kindof(\Gamma, X::K \vdash T :: ?) \ \textrm{in} $ \\
$\qquad\qquad (\sigma, \sigma K \Rightarrow K_T) $ \\
\end{tabular} \\
\begin{tabular}{l}
$kindof(\Gamma \vdash T_1 \; T_2 :: ?) = $ \\
$\qquad \textrm{let} \ (\sigma_1, K_1) = kindof(\Gamma \vdash T_1 :: ?) \ \textrm{in} $ \\
$\qquad \textrm{let} \ (\sigma_2, K_2) = kindof(\sigma_1 \Gamma \vdash \sigma_1 T_2 :: ?) \ \textrm{in} $ \\
$\qquad \textrm{let} \ \widehat{X} = fresh \ \textrm{in} $ \\
$\qquad \textrm{let} \ \sigma_3 = unify(\sigma_2 K_1, K_2 \Rightarrow \widehat{X}) \ \textrm{in} $ \\
$\qquad\qquad (\sigma_3 \circ \sigma_2 \circ \sigma_1, \sigma_3 \widehat{X}) $ \\
\end{tabular} \\
\begin{tabular}{l}
$kindof(\Gamma \vdash \bar{X} :: ?) = $ \\
$\qquad\qquad ([], *) $ \\
\end{tabular} \\

Dla typów:\\
\begin{tabular}{l}
$typeof(\Gamma \vdash x : ?) = $ \\
$\qquad \textrm{let} \ \Omega \bar{X}_1 \dots \bar{X}_n \widehat{X}_1 \dots \widehat{X}_m.T = \Gamma(x) \ \textrm{in} $ \\
$\qquad \textrm{let} \ \bar{Y}_1, \dots, \bar{Y}_n, \widehat{Y}_1, \dots, \widehat{Y}_m = fresh \ \textrm{in} $ \\
$\qquad\qquad ([], [\bar{X}_1 := \bar{Y}_1, \dots ,\bar{X}_n := \bar{Y}_n,\widehat{X}_1 := \widehat{Y}_1, \dots ,\widehat{X}_m := \widehat{Y}_m]T) $ \\
\end{tabular} \\
\begin{tabular}{l}
$typeof(\Gamma \vdash \lambda x:T.t : ?) = $ \\
$\qquad \textrm{let} \ (\sigma_1, K) = kindof(\Gamma \vdash T :: ?) \ \textrm{in} $ \\
$\qquad \textrm{let} \ \sigma_2 = unify(K, *) \ \textrm{in} $ \\
$\qquad \textrm{let} \ (\sigma_3, T_t) = typeof(\sigma_2 \sigma_1(\Gamma, x:T) \vdash \sigma_2 \sigma_1 t : ?) \ \textrm{in} $ \\
$\qquad\qquad (\sigma_3 \circ \sigma_2 \circ \sigma_1, \sigma_3 \sigma_2 \sigma_1 T \rightarrow T_t) $ \\
\end{tabular} \\
\begin{tabular}{l}
$typeof(\Gamma \vdash t_1 \; t_2 : ?) = $ \\
$\qquad \textrm{let} \ (\sigma_1, T_1) = typeof(\Gamma \vdash t_1 : ?) \ \textrm{in} $ \\
$\qquad \textrm{let} \ (\sigma_2, T_2) = typeof(\sigma_1 \Gamma \vdash \sigma_1 t_2 : ?) \ \textrm{in} $ \\
$\qquad \textrm{let} \ \bar{X} = fresh \ \textrm{in} $ \\
$\qquad \textrm{let} \ (\sigma_3, T'_2 \to T_3) = unify_\beta(\sigma_2 T_1, T_2 \rightarrow \bar{X}) \ \textrm{in} $ \\
$\qquad\qquad (\sigma_3 \circ \sigma_2 \circ \sigma_1, T_3) $ \\
\end{tabular} \\
\begin{tabular}{l}
$typeof(\Gamma \vdash \Delta X::K.t : ?) = $ \\
$\qquad \textrm{let} \ (\sigma, T) = typeof(\Gamma, X::K \vdash t : ?) \ \textrm{in} $ \\
$\qquad\qquad (\sigma \setminus X, \forall X::\sigma K.T) $ \\
\end{tabular} \\
\begin{tabular}{l}
$typeof(\Gamma \vdash t[T] : ?) = $ \\
$\qquad \textrm{let} \ (\sigma_1, T_t) = typeof(\Gamma \vdash t : ?) \ \textrm{in} $ \\
$\qquad \textrm{let} \ (\sigma_2, K) = kindof(\sigma_1 \Gamma \vdash \sigma_1 T :: ?) \ \textrm{in} $ \\
$\qquad \textrm{let} \ \bar{X} = fresh \ \textrm{in} $ \\
$\qquad \textrm{let} \ Y = fresh \ \textrm{in} $ \\
$\qquad \textrm{let} \ (\sigma_3, \forall Z :: K' . T') = unify_\beta(\sigma_2 T_t, \forall Y::K.\bar{X}) \textrm{in} $ \\
$\qquad\qquad (\sigma_3 \circ \sigma_2 \circ \sigma_1, \{Y:=\sigma_3 \sigma_2 \sigma_1 T\} T') $ \\
\end{tabular} \\
\begin{tabular}{l}
$typeof(\Gamma \vdash \textrm{let} \ x=t_1 \ \textrm{in} \ t_2 : ?) = $ \\
$\qquad \textrm{let} \ (\sigma_1, T_1) = typeof(\Gamma \vdash t_1 : ?) \ \textrm{in} $ \\
$\qquad \textrm{let} \ \left\{ \bar{X}_1, \dots, \bar{X}_n \right\} = \bar{Var}(T_1) \setminus \bar{Var}(\Gamma) \ \textrm{in} $ \\
$\qquad \textrm{let} \ \left\{ \widehat{X}_1, \dots, \widehat{X}_m \right\} = \widehat{Var}(T_1) \setminus \widehat{Var}(\Gamma) \ \textrm{in} $ \\
$\qquad \textrm{let} \ (\sigma_2, T_2) = typeof(\sigma_1 \Gamma, x:\Omega \bar{X}_1 \dots \bar{X}_n \widehat{X}_1 \dots \widehat{X}_m. T_2 \vdash \sigma_1 t_2 : ?) \ \textrm{in} $ \\
$\qquad\qquad (\sigma_2 \circ \sigma_1, T_2) $ \\
\end{tabular} \\
\begin{tabular}{l}
$typeof(\Gamma \vdash \textrm{tlet} \ X=T \ \textrm{in} \ t : ?) = $ \\
$\qquad \textrm{let} \ (\sigma_1, K) = kindof(\Gamma \vdash T :: ?) \ \textrm{in} $ \\
$\qquad \textrm{let} \ (\sigma_2, T_t) = typeof(\sigma_1 \Gamma \vdash \{ X(\widehat{Var}(K) \setminus \widehat{Var}(\Gamma)) := \sigma_1 T \} \sigma_1 t : ?) \ \textrm{in} $ \\
$\qquad\qquad (\sigma_2 \circ \sigma_1, T_t) $ \\
\end{tabular} \\

\subsection{Własności}

\begin{twierdzenie}
Funkcja $typeof$ zatrzymuje się.
\end{twierdzenie}
\begin{proof}
\end{proof}

\begin{twierdzenie}
Niech $t$ i~$\Gamma$ nie zawierają zmiennych schematowych. Wówczas \mbox{$typeof(\Gamma \vdash t : ?) = (\sigma, T)$} wtedy 
i~tylko wtedy gdy $\Gamma \vdash t : T'$, gdzie $T' =_\beta T$ oraz $\sigma = []$ (tzn. gdy program zawiera wszystkie anotacje, algorytm $typeof$ wykonuje zwykłą 
kontrolę typów w~systemie $F_\omega$).
\label{twtypecheck}
\end{twierdzenie}
\begin{proof}
Szkic dowodu:
\begin{enumerate}
\item Indukcyjnie po strukturze typu pokazujemy analogiczne twierdzenie dla rodzajów i~funkcji $kindof$:\\
Niech $T$ i~$\Gamma$ nie zawierają zmiennych schematowych. Wówczas \mbox{$kindof(\Gamma \vdash t :: ?) = (\sigma, K)$} wtedy
i~tylko wtedy gdy $\Gamma \vdash T :: K$ oraz $\sigma = []$.
\item Indukcyjnie po strukturze wyrażenia $t$ pokażemy tezę implikację tezy w~jedną stronę.
\item Pokazujemy indukcyjnie, że drzewa wyprowadzenia $\Gamma \vdash t : T$ można zawsze przebudować tak, by
z~regóły
\[
\frac{\Gamma \vdash t : S \quad S \equiv T \quad \Gamma \vdash T :: *}{\Gamma \vdash t : T}
\]
korzystać jedynie nad argumentem aplikacji i~w~kożeniu drzewa wyprowadzenia.
\item Indukcyjnie po tak przebudowanym drzewie pokażemy implikację tezy w~drugą stronę.
\end{enumerate}
\end{proof}

\begin{twierdzenie}
Niech $t$ i~$\Gamma$ będą takim termem i~kontekstem, że $typeof(\Gamma \vdash t : ?) = (\sigma, T)$. Wówczas
istnieje takie podstawienie schematowe $\theta$, że zachodzą następujące własności:
\begin{itemize}
\item $\theta \Gamma \vdash \theta t : T$
\item $\theta \setminus BTV(t) = \sigma$
\end{itemize}
\end{twierdzenie}
\begin{proof}
Szkic dowodu:
\begin{enumerate}
\item Podobnie jak w~dowodzie twierdzenia \ref{twtypecheck} najpierw indukcyjnie dowodzimy analogicznego twierdzenia
dla funkcji $kindod$ i~relacji rodzajowania.
\item Indukcyjnie po strukturze termu $t$ pokazujemy tezę. Podstawienie schematowe $\theta$ konstruujemy tak jak podstawienie schematowe
$\sigma$ tyle, że z~pominięciem alnulowania zmiennych związanych. W~miejscach w~których korzystamy z~$\beta$-unifikacji
korzystamy z~definicji \mbox{$\beta$-unifikatora} by zkonstruować podstawienie schematowe $\theta$.
\end{enumerate}
\end{proof}

\end{document}
