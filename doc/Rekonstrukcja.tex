\documentclass[11pt,leqno]{article}

\usepackage[polish]{babel}
\usepackage[utf8]{inputenc}
\usepackage[T1]{fontenc}


\usepackage{a4wide}

\usepackage{amsfonts}
\usepackage{amsmath}
\usepackage[pdftex]{graphicx}
\usepackage{caption}
\usepackage{bm}
\usepackage{amsthm}
\usepackage{indentfirst}

 
%%%%%%%%%%%%%%%%%%
% Kropka po numerze paragrafu, podparagrafu itp. 

\makeatletter
 \renewcommand\@seccntformat[1]{\csname the#1\endcsname.\quad}
 \renewcommand\numberline[1]{#1.\hskip0.7em}
\makeatother

%%%%%%%%%%%%%%%%%%
% Kropka po numerze tablicy, rysunku i ustawienie czcionki dla etykiety. 

\captionsetup{labelfont=sc,labelsep=period}

%%%%%%%%%%%%%%%%%%
% Inna numeracja wzorów.

\renewcommand{\theequation}{\arabic{section}.\arabic{equation}}

%%%%%%%%%%%%%%%%%%

\title{{\textbf{Rekonstrukcja typów w systemie $F_{\omega}$}}\\[1ex]}
\date{Wrocław, dnia \today\ r.}
\author{Piotr Polesiuk \\ \texttt{bassists@o2.pl}}

\begin{document}
\thispagestyle{empty}
\maketitle

\pagebreak

\theoremstyle{plain}
\newtheorem{twierdzenie}{Twierdzenie}
\newtheorem{lemat}{Lemat}

\theoremstyle{definition}
\newtheorem{definicja}{Definicja}

%%%%%%%%%%%%%%%%%%%%%%%%%%%%%%%%%%%%%%%%%%%%%%%%%%%%%%%%%%%%%%%%%%%%%%%%%%%%%%%%
%%%%%%%%%%%%%%%%%%%%%%%%%%%%%%%%%%%%%%%%%%%%%%%%%%%%%%%%%%%%%%%%%%%%%%%%%%%%%%%%
\section{Kilka słów o składni.}
%%%%%%%%%%%%%%%%%%%%%%%%%%%%%%%%%%%%%%%%%%%%%%%%%%%%%%%%%%%%%%%%%%%%%%%%%%%%%%%%
%%%%%%%%%%%%%%%%%%%%%%%%%%%%%%%%%%%%%%%%%%%%%%%%%%%%%%%%%%%%%%%%%%%%%%%%%%%%%%%%
\setcounter{equation}{0}

Rozważmy język o składni:

\begin{tabular}{| l c r |}
  \hline
  t ::= &  &  \\
   & $x$ & \textit{zmienne}  \\
   & $\lambda x:T.t$ & \textit{abstrakcja (anotowana)} \\
   & $\lambda x.t$ & \textit{abstrakcja (nieanotowana)} \\
   & $t\;t$ & \textit{aplikacja} \\
   & $\Delta X::K.t$ & \textit{abstrakcja typowa (anotowana)} \\
   & $\Delta X.t$ & \textit{abstrakcja typowa (nieanotowana)} \\
   & $t[T]$ & \textit{aplikacja typowa} \\
   & $\textrm{let} \ x = t \ \textrm{in} \ t$ & \textit{zmienna lokalna} \\
   & $\textrm{tlet} \ X = T \ \textrm{in} \ t$ & \textit{typ lokalny} \\
   & & \\
  T ::= &  &  \\
   & $X$ & \textit{zmienna typowa (kwantyfikowana abstrakcją)} \\
   & $T \rightarrow T$ & \textit{typ funkcji} \\
   & $\forall X::K.T$ & \textit{typ uniwersalny (anotowany)} \\
   & $\forall X.T$ & \textit{typ uniwersalny (nieanotowany)} \\
   & $\Lambda X::K.T$ & \textit{konstruktor abstrakcji typowej (anotowany)} \\
   & $\Lambda X.T$ & \textit{konstruktor abstrakcji typowej (nieanotowany)} \\
   & $T\;T$ & \textit{aplikacja konstruktora typowego} \\
   & $\bar{X}$ & \textit{zmienna typowa (kwantyfikowana schematem)} \\
   & & \\
  K ::= & & \\
   & $*$ & \textit{rodzaj wszystkich typów} \\
   & $K \Rightarrow K$ & \textit{rodzaj funkcji typowej} \\
   & $\widehat{X}$ & \textit{zmienna rodzajowa (kwantyfikowana schematem)} \\
   & & \\
  \hline
\end{tabular} \\

Ważnym założeniem jest to, że $\bar{X}::*$, co nam ułatwi sprawę przy unifikacji, a~zarazem zbytnio nie ograniczy języka.
  
Jeżeli chcemy dodać ML-polimorfizm, będą nam portrzebne schematy typów i schematy rodzajów:

\begin{tabular}{| l c r |}
  \hline
  $\left<T\right>$ ::= &  & \textit{schemat typu} \\
   & $T$ & \textit{typ schematu} \\
   & $\Omega \bar{X}.\left<T\right>$ & \textit{kwantyfikacja zmiennej typowej} \\
   & $\Omega \widehat{X}.\left<T\right>$ & \textit{kwantyfikacja zmiennej rodzajowej} \\
   & & \\
  $\left<K\right>$ ::= &  & \textit{schemat rodzaju} \\
   & $K$ & \textit{rodzaj schematu}\\
   & $\Omega \widehat{X}.\left<K\right>$ & \textit{kwantyfikacja zmiennej rodzajowej} \\
   & & \\
  \hline
\end{tabular} \\

Będziemy używać następującego cukru syntaktycznego:
\[
\Omega \bar{X}_1 \dots \bar{X}_n \widehat{X}_1 \dots \widehat{X}_m.T \equiv 
\Omega \bar{X}_1. \dots \Omega \bar{X}_n. \Omega \widehat{X}_1. \dots \Omega \widehat{X}_m.T
\]
I analogicznie dla rodzajów.

Teraz sam kontekst typowania ma postać:

\begin{tabular}{| l c r |}
  \hline
  $\Gamma$ ::= &  & \\
   & $\emptyset$ & \\
   & $\Gamma, x:\left<T\right>$ & \\
   & $\Gamma, X::\left<K\right>$ & \\
   & & \\
  \hline
\end{tabular} \\

\section{$\beta$-unifikacja}

Podczas rekonstukcji typów pojawiają się równania więzów które należy rozwiązać. Samo rozwiązanie
sprowadza się do unifikacji pewnych termów, ale tutaj, ze względu na możliwość występowania funkcji
typowych, termy równoważne nie muszą być równe, więc sama unifikacja powinna sprowadzać termy do
$\beta$-równych sobie.

Taka unifikacja niesie ze sobą wiele problemów. Po pierwsze, będziemy chcieli używać podstawienia,
będącego unifikatorem, na innych termach. To zaś grozi uzewnętrznieniem zmiennych związanych. Drugi
problem to taki, że samo podstawienie nie musi zrównywać termów, które da się zrównać, ale różnią się
nazwami zmiennych związanych.

Z pierwszym problemem poradzimy sobie, traktując podstawienia jako funkcję częściowe, tzn. samo podstawienie
oprócz przyporzątkowań postaci $[\bar{X} := T]$, może również zawierać podstawienia postaci $[\bar{X} := fail]$.
Oraz dodając jeszcze operację anulowania podstawienia:
\[
[\bar{X} := T] \setminus X = 
\begin{cases}
[\bar{X} := fail] & X \in    Var(T) \\
[\bar{X} := T]    & X \notin Var(T)
\end{cases}
\]

Z drugim problemem poradzimy sobie traktując unifikator jako parę zawierającą podstawienie, które dobrze działa
na zewnątrz, oraz term będący wynikiem unifikacji.

\begin{definicja}
\emph{$\beta$-unifikatorem} dla konstruktorów typów $T_1$ i $T_2$ nazwiemy taką parę $(\sigma, S)$, że istnieją
podstawienia $\theta_1$ i $\theta_2$ takie, że:
\[
\theta_1 T_1 =_\beta S =_\beta \theta_2 T_2
\]
oraz
\[
\theta_1 \setminus BTV(T_1) = \sigma = \theta_2 \setminus BTV(T_2)
\]
\end{definicja}

\begin{definicja}
Powiemy, że $\beta$-unifikator $(\sigma, S)$ jest \emph{ogólniejszy} od $\beta$-unifikatora
$(\theta, T)$, jeżeli istnieje takie podstawienie $\rho$ i przemianowanie $\alpha$, że
\[
\rho\alpha S =_\beta T \quad \textrm{oraz} \quad \rho\alpha\sigma = \theta
\]
\end{definicja}

\begin{definicja}
\emph{Najogólniejszym $\beta$-unifikatorem} dla konstruktorów typów $T_1$ i $T_2$ nazwiemy taki $\beta$-unifikator,
który jest ogólniejszy od wszystkich innych $\beta$-unifikatorów tychże konstruktorów typów.
\end{definicja}

\subsection{Algorytm $\beta$-unifikacji}

Przy unifikacji wszystkie nieanotowane kwantyfikatory traktujemy jak anotowane unikatową zmienną.
Funkcja $unify$ to klasyczna unifikacja (na rodzajach).\\
\begin{tabular}{l}
$unify_\beta(T_1, T_2) \ \textrm{when} \ T_1 =_\beta T_2 \ = $ \\
$\qquad ([], T_1)$ \\
\end{tabular} \\
\begin{tabular}{l}
$unify_\beta(\bar{X}, T) = $ \\
$\qquad \textrm{if} \ \bar{X} \in \bar{Var}(T) \ \textrm{then} \ fail $ \\
$\qquad \textrm{else} \ ([\bar{X} := T], T) $ \\
\end{tabular} \\
\begin{tabular}{l}
$unify_\beta(T, \bar{X}) = $ \\
$\qquad \textrm{if} \ \bar{X} \in \bar{Var}(T) \ \textrm{then} \ fail $ \\
$\qquad \textrm{else} \ ([\bar{X} := T], T) $ \\
\end{tabular} \\
\begin{tabular}{l}
$unify_\beta(\forall X_1::K_1.T_1, \forall X_2::K_2.T_2) = $ \\
$\qquad \textrm{let} \ \sigma_K = unify(K_1, K_2) \ \textrm{in} $ \\
$\qquad \textrm{let} \ (\sigma_T, T) = unify_\beta(\sigma_K T_1, \sigma_K \{X_2 := X_1\}T_2) \ \textrm{in} $ \\
$\qquad\qquad ((\sigma_T \setminus X_1) \circ \sigma_K, \forall X_1::\sigma_K K_1.T) $ \\
\end{tabular} \\
\begin{tabular}{l}
$unify_\beta(\Lambda X_1::K_1.T_1, \Lambda X_2::K_2.T_2) = $ \\
$\qquad \textrm{let} \ \sigma_K = unify(K_1, K_2) \ \textrm{in} $ \\
$\qquad \textrm{let} \ (\sigma_T, T) = unify_\beta(\sigma_K T_1, \sigma_K \{X_2 := X_1\}T_2) \ \textrm{in} $ \\
$\qquad\qquad ((\sigma_T \setminus X_1) \circ \sigma_K, \Lambda X_1::\sigma_K K_1.T) $ \\
\end{tabular} \\
\begin{tabular}{l}
$unify_\beta(T_1 \rightarrow S_1, T_2 \rightarrow S_2) = $ \\
$\qquad \textrm{let} \ (\theta, T) = unify_\beta(T_1, T_2) \ \textrm{in} $ \\
$\qquad \textrm{let} \ (\sigma, S) = unify_\beta(\theta S_1, \theta S_2) \ \textrm{in} $ \\
$\qquad\qquad (\sigma \circ \theta, \sigma T \rightarrow S) $ \\
\end{tabular} \\
\begin{tabular}{l}
$unify_\beta(T_1, T_2) \ \textrm{when} \ T_1 \longrightarrow_\beta S_1 \ = $ \\
$\qquad unify_\beta(S_1, T_2)$ \\
\end{tabular} \\
\begin{tabular}{l}
$unify_\beta(T_1, T_2) \ \textrm{when} \ T_2 \longrightarrow_\beta S_2 \ = $ \\
$\qquad unify_\beta(T_1, S_2)$ \\
\end{tabular} \\
\begin{tabular}{l}
$unify_\beta(V_1 \; T_1, V_2 \; T_2) = $ \\
$\qquad \textrm{let} \ (\theta, V) = unify_\beta(V_1, V_2) \ \textrm{in} $ \\
$\qquad \textrm{let} \ (\sigma, T) = unify_\beta(\theta T_1, \theta T_2) \ \textrm{in} $ \\
$\qquad\qquad (\sigma \circ \theta, \sigma V \; T) $ \\
\end{tabular} \\

Tu powinno pojawić się kilka dowodów własności.

\section{Rekonstrukcja typów}

\subsection{Algorytm W}

Dla rodzajów: \\
\begin{tabular}{l}
$kindof(\Gamma \vdash X :: ?) = $ \\
$\qquad \textrm{let} \ \Omega \widehat{X}_1 \dots \widehat{X}_n.K = \Gamma(X) \ \textrm{in} $ \\
$\qquad \textrm{let} \ \widehat{Y}_1, \dots, \widehat{Y}_n = fresh \ \textrm{in} $ \\
$\qquad\qquad ([], [\widehat{X}_1 := \widehat{Y}_1, \dots ,\widehat{X}_n := \widehat{Y}_n]K) $ \\
\end{tabular} \\
\begin{tabular}{l}
$kindof(\Gamma \vdash T_1 \rightarrow T_2 :: ?) = $ \\
$\qquad \textrm{let} \ (\sigma_1, K_1) = kindof(\Gamma \vdash T_1 :: ?) \ \textrm{in} $ \\
$\qquad \textrm{let} \ (\sigma_2, K_2) = kindof(\sigma_1 \Gamma \vdash \sigma_1 T_1 :: ?) \ \textrm{in} $ \\
$\qquad \textrm{let} \ \rho_1 = unify(\sigma_2 K_1, *) \ \textrm{in} $ \\
$\qquad \textrm{let} \ \rho_2 = unify(\rho_1 K_2, *) \ \textrm{in} $ \\
$\qquad\qquad (\rho_2 \circ \rho_1 \circ \sigma_2 \circ \sigma_1, *) $ \\
\end{tabular} \\
\begin{tabular}{l}
$kindof(\Gamma \vdash \forall X::K.T :: ?) = $ \\
$\qquad \textrm{let} \ (\sigma, K_T) = kindof(\Gamma, X::K \vdash T :: ?) \ \textrm{in} $ \\
$\qquad \textrm{let} \ \rho = unify(K_T, *) \ \textrm{in} $ \\
$\qquad\qquad (\rho \circ \sigma, *) $ \\
\end{tabular} \\
\begin{tabular}{l}
$kindof(\Gamma \vdash \Lambda X::K.T :: ?) = $ \\
$\qquad \textrm{let} \ (\sigma, K_T) = kindof(\Gamma, X::K \vdash T :: ?) \ \textrm{in} $ \\
$\qquad\qquad (\sigma, K \Rightarrow K_T) $ \\
\end{tabular} \\
\begin{tabular}{l}
$kindof(\Gamma \vdash T_1 \; T_2 :: ?) = $ \\
$\qquad \textrm{let} \ (\sigma_1, K_1) = kindof(\Gamma \vdash T_1 :: ?) \ \textrm{in} $ \\
$\qquad \textrm{let} \ (\sigma_2, K_2) = kindof(\sigma_1 \Gamma \vdash \sigma_1 T_2 :: ?) \ \textrm{in} $ \\
$\qquad \textrm{let} \ \widehat{X} = fresh \ \textrm{in} $ \\
$\qquad \textrm{let} \ \sigma_3 = unify(\sigma_2 K_1, K_2 \Rightarrow \widehat{X}) \ \textrm{in} $ \\
$\qquad\qquad (\sigma_3 \circ \sigma_2 \circ \sigma_1, \sigma_3 \widehat{X}) $ \\
\end{tabular} \\
\begin{tabular}{l}
$kindof(\Gamma \vdash \bar{X} :: ?) = $ \\
$\qquad\qquad ([], *) $ \\
\end{tabular} \\

Dla typów:\\
\begin{tabular}{l}
$typeof(\Gamma \vdash x : ?) = $ \\
$\qquad \textrm{let} \ \Omega \bar{X}_1 \dots \bar{X}_n \widehat{X}_1 \dots \widehat{X}_m.T = \Gamma(x) \ \textrm{in} $ \\
$\qquad \textrm{let} \ \bar{Y}_1, \dots, \bar{Y}_n, \widehat{Y}_1, \dots, \widehat{Y}_m = fresh \ \textrm{in} $ \\
$\qquad\qquad ([], [\bar{X}_1 := \bar{Y}_1, \dots ,\bar{X}_n := \bar{Y}_n,\widehat{X}_1 := \widehat{Y}_1, \dots ,\widehat{X}_m := \widehat{Y}_m]T) $ \\
\end{tabular} \\
\begin{tabular}{l}
$typeof(\Gamma \vdash \lambda x:T.t : ?) = $ \\
$\qquad \textrm{let} \ (\sigma_1, K) = kindof(\Gamma \vdash T :: ?) \ \textrm{in} $ \\
$\qquad \textrm{let} \ \sigma_2 = unify(K, *) \ \textrm{in} $ \\
$\qquad \textrm{let} \ (\sigma_3, T_t) = typeof(\sigma_2 \sigma_1(\Gamma, x:T) \vdash \sigma_2 \sigma_1 t : ?) \ \textrm{in} $ \\
$\qquad\qquad (\sigma_3 \circ \sigma_2 \circ \sigma_1, \sigma_3 \sigma_2 \sigma_1 T \rightarrow T_t) $ \\
\end{tabular} \\
\begin{tabular}{l}
$typeof(\Gamma \vdash t_1 \; t_2 : ?) = $ \\
$\qquad \textrm{let} \ (\sigma_1, T_1) = typeof(\Gamma \vdash t_1 : ?) \ \textrm{in} $ \\
$\qquad \textrm{let} \ (\sigma_2, T_2) = typeof(\sigma_1 \Gamma \vdash \sigma_1 t_2 : ?) \ \textrm{in} $ \\
$\qquad \textrm{let} \ \bar{X} = fresh \ \textrm{in} $ \\
$\qquad \textrm{let} \ (\sigma_3, T_3) = unify_\beta(\sigma_2 T_1, T_2 \rightarrow \bar{X}) \ \textrm{in} $ \\
$\qquad\qquad (\sigma_3 \circ \sigma_2 \circ \sigma_1, \sigma_3 \bar{X}) $ \\
\end{tabular} \\
\begin{tabular}{l}
$typeof(\Gamma \vdash \Delta X::K.t : ?) = $ \\
$\qquad \textrm{let} \ (\sigma, T) = typeof(\Gamma, X::K \vdash t : ?) \ \textrm{in} $ \\
$\qquad\qquad (\sigma \setminus X, \forall X::\sigma K.T) $ \\
\end{tabular} \\
\begin{tabular}{l}
$typeof(\Gamma \vdash t[T] : ?) = $ \\
$\qquad \textrm{let} \ (\sigma_1, T_t) = typeof(\Gamma \vdash t : ?) \ \textrm{in} $ \\
$\qquad \textrm{let} \ (\sigma_2, K) = kindof(\sigma_1 \Gamma \vdash \sigma_1 T :: ?) \ \textrm{in} $ \\
$\qquad \textrm{let} \ \bar{X} = fresh \ \textrm{in} $ \\
$\qquad \textrm{let} \ Y = fresh \ \textrm{in} $ \\
$\qquad \textrm{let} \ (\sigma_3, T'_t) = unify_\beta(\sigma_2 T_t, \forall Y::K.\bar{X}) \textrm{in} $ \\
$\qquad\qquad (\sigma_3 \circ \sigma_2 \circ \sigma_1, [Y:=\sigma_3 \sigma_2 \sigma_1 T]\sigma_3\bar{X}) $ \\
\end{tabular} \\
\begin{tabular}{l}
$typeof(\Gamma \vdash \textrm{let} \ x=t_1 \ \textrm{in} \ t_2 : ?) = $ \\
$\qquad \textrm{let} \ (\sigma_1, T_1) = typeof(\Gamma \vdash t_1 : ?) \ \textrm{in} $ \\
$\qquad \textrm{let} \ \left\{ \bar{X}_1, \dots, \bar{X}_n \right\} = \bar{FTV}(T_1) \setminus \bar{FTV}(\Gamma) \ \textrm{in} $ \\
$\qquad \textrm{let} \ \left\{ \widehat{X}_1, \dots, \widehat{X}_m \right\} = \widehat{FKV}(T_1) \setminus \widehat{FKV}(\Gamma) \ \textrm{in} $ \\
$\qquad \textrm{let} \ (\sigma_2, T_2) = typeof(\sigma_1 \Gamma, x:\Omega \bar{X}_1 \dots \bar{X}_n \widehat{X}_1 \dots \widehat{X}_m. T_2 \vdash \sigma_1 t_2 : ?) \ \textrm{in} $ \\
$\qquad\qquad (\sigma_2 \circ \sigma_1, T_2) $ \\
\end{tabular} \\
\begin{tabular}{l}
$typeof(\Gamma \vdash \textrm{tlet} \ X=T \ \textrm{in} \ t : ?) = $ \\
$\qquad \textrm{let} \ (\sigma_1, K) = kindof(\Gamma \vdash T :: ?) \ \textrm{in} $ \\
$\qquad \textrm{let} \ \left\{ \widehat{X}_1, \dots, \widehat{X}_m \right\} = \widehat{FKV}(K) \setminus \widehat{FKV}(\Gamma) \ \textrm{in} $ \\
$\qquad \textrm{let} \ (\sigma_2, T_t) = typeof(\sigma_1 \Gamma, X::\Omega \widehat{X}_1 \dots \widehat{X}_m.K \vdash \sigma_1 t : ?) \ \textrm{in} $ \\
$\qquad\qquad (\sigma_2 \circ \sigma_1, T_t) $ \\
\end{tabular} \\

Tu można wspomnieć o jego własnościach.

\end{document}

