%%%%%%%%%%%%%%%%%%%%%%%%%%%%%%%%%%%%%%%%%%%%%%%%%%%%%%%%%%%%%%%%%%%%%%%%%%%%%%%%
%%%%%%%%%%%%%%%%%%%%%%%%%%%%%%%%%%%%%%%%%%%%%%%%%%%%%%%%%%%%%%%%%%%%%%%%%%%%%%%%
%%                                                                            %%
%%                                              Wrocław, styczeń 2011 r.      %%
%%                                                                            %%
%%                                                                            %%
%%                                                                            %%
%%                                SYSTEMY TYPÓW - PROJEKT                     %%
%%                                                                            %%
%%                                   Piotr Polesiuk                           %%
%%                                Małgorzata Jurkiewicz                       %%
%%                                                                            %%
%%                                                                            %%
%%%%%%%%%%%%%%%%%%%%%%%%%%%%%%%%%%%%%%%%%%%%%%%%%%%%%%%%%%%%%%%%%%%%%%%%%%%%%%%%
%%%%%%%%%%%%%%%%%%%%%%%%%%%%%%%%%%%%%%%%%%%%%%%%%%%%%%%%%%%%%%%%%%%%%%%%%%%%%%%%


\documentclass[11pt,leqno]{article}

\usepackage[polish]{babel}
\usepackage[utf8]{inputenc}
\usepackage[T1]{fontenc}


\usepackage{a4wide}

\usepackage{amsfonts}
\usepackage{amsmath}
\usepackage[pdftex]{graphicx}
\usepackage{caption}
\usepackage{bm}
\usepackage{amsthm}

 
%%%%%%%%%%%%%%%%%%
% Kropka po numerze paragrafu, podparagrafu itp. 

\makeatletter
 \renewcommand\@seccntformat[1]{\csname the#1\endcsname.\quad}
 \renewcommand\numberline[1]{#1.\hskip0.7em}
\makeatother

%%%%%%%%%%%%%%%%%%
% Kropka po numerze tablicy, rysunku i ustawienie czcionki dla etykiety. 

\captionsetup{labelfont=sc,labelsep=period}

%%%%%%%%%%%%%%%%%%
% Inna numeracja wzorów.

\renewcommand{\theequation}{\arabic{section}.\arabic{equation}}

%%%%%%%%%%%%%%%%%%

\title{{\textbf{System typów $F_{\omega}$}}\\[1ex]
       {\Large Systemy Typów 2010/11}\\[-1ex]
       {\large Prowadzący: dr Dariusz Biernacki}}
\author{Piotr Polesiuk \\ \texttt{bassists@o2.pl} 
   \and Małgorzata Jurkiewicz \\ \texttt{gosia.jurkiewicz@gmail.com}} 

\date{Wrocław, dnia \today\ r.}

\begin{document}
\thispagestyle{empty}
\maketitle

\pagebreak

\theoremstyle{plain}
\newtheorem{twierdzenie}{Twierdzenie}
\newtheorem{lemat}{Lemat}
\newtheorem{fakt}{Fakt}

\theoremstyle{definition}
\newtheorem{definicja}{Definicja}



%%%%%%%%%%%%%%%%%%%%%%%%%%%%%%%%%%%%%%%%%%%%%%%%%%%%%%%%%%%%%%%%%%%%%%%%%%%%%%%%
%%%%%%%%%%%%%%%%%%%%%%%%%%%%%%%%%%%%%%%%%%%%%%%%%%%%%%%%%%%%%%%%%%%%%%%%%%%%%%%%
\section{Wstęp}
%%%%%%%%%%%%%%%%%%%%%%%%%%%%%%%%%%%%%%%%%%%%%%%%%%%%%%%%%%%%%%%%%%%%%%%%%%%%%%%%
%%%%%%%%%%%%%%%%%%%%%%%%%%%%%%%%%%%%%%%%%%%%%%%%%%%%%%%%%%%%%%%%%%%%%%%%%%%%%%%%
\setcounter{equation}{0}


Tematem naszej pracy jest system typów $F_\omega$. Praca powstała jako część teoretyczna do 
projektu zaliczeniowego z systemu typów prowadzonego na Uniwersytecie Wrocławskim w roku akademickim
2010/2011. W pracy przedstawiliśmy ten system głównie od strony teoretycznej, zarówno minimalną część
$F_\omega$, jak i jego rozszerzenia,  ale zaprezentowaliśmy również stworzony przez nas algorytm 
rekonstrukcji typów, choć rekonstrukcja typów w systemie $F_\omega$ jest 
nierozstrzygalna. Wady i własności tego algorytmu również przedstawiliśmy w pracy.
Część praktyczna do projektu została napisana w języku F\# i jest załączona do pracy.



%%%%%%%%%%%%%%%%%%%%%%%%%%%%%%%%%%%%%%%%%%%%%%%%%%%%%%%%%%%%%%%%%%%%%%%%%%%%%%%%
%%%%%%%%%%%%%%%%%%%%%%%%%%%%%%%%%%%%%%%%%%%%%%%%%%%%%%%%%%%%%%%%%%%%%%%%%%%%%%%%
\section{System $F_{\omega}$}                                 
%%%%%%%%%%%%%%%%%%%%%%%%%%%%%%%%%%%%%%%%%%%%%%%%%%%%%%%%%%%%%%%%%%%%%%%%%%%%%%%%
%%%%%%%%%%%%%%%%%%%%%%%%%%%%%%%%%%%%%%%%%%%%%%%%%%%%%%%%%%%%%%%%%%%%%%%%%%%%%%%%
\setcounter{equation}{0}

W rozdziale tym chcielibyśmy się skupić na systemie $F_{\omega}$ okrojonym do niezbędnego minimum.
Przedstawimy, jak wyglądają termy, typy i wartości tego języka, a także pokażemy, jak przebiega typowanie,
znajdowanie rodzaju, ewaluacja czy sprawdzanie równości typów. Postaramy się pisać jasno i pokażemy parę przykładów, 
aby nieobyty w temacie Czytelnik nie zgubił się.  W rozdziale trzecim do tak zdefiniowanego systemu będziemy wprowadzać
rozszerzenia.

\subsection {Termy i typy w $F_\omega$}

System $F_{\omega}$ to rachunek będący rozszerzeniem $\lambda_{\omega}$ oraz systemu $F$.
Wszystkie trzy wywodzą się z rachunku lambda z typami prostymi. Termy oraz typy definiujemy w $\lambda_{\rightarrow}$ następująco: \\ \\
\begin{tabular}{| l c r |}
  \hline
  $\mathtt{t}$ ::= &  & \textit{termy}  \\
   & $\mathtt{x}$ & \textit{zmienne}  \\
   & $\mathtt{\lambda x:T.t}$ & \textit{abstrakcja anotowana} \\
   & $\mathtt{\lambda x.t}$ & \textit{abstrakcja nieanotowana} \\
   & $\mathtt{t \hspace{0.2em} t}$ & \textit{aplikacja} \\
   & & \\
  $\mathtt{T}$ ::= &  & \textit{typy} \\
   & $\mathtt{X}$ & \textit{zmienna typowa} \\
   & $\mathtt{T \rightarrow T}$ & \textit{typ funkcji} \\
  \hline
\end{tabular} \\



\subsubsection{System $\lambda_{\omega}$} 

Główną cechą systemu $\lambda_{\omega}$ jest to, że oprócz termów zależnych od termów mamy typy zależne od typów, czyli możemy mówić o aplikacji i abstrakcji typowej, a tak powstałe 'typy' będziemy nazywać konstruktorami. By nam się nie pomyliło z abstrakcją na termach, zmienne konstruktorowe będziemy zaczynać dużą literą. Przykładowo $\mathtt{Tb=\lambda X.X \rightarrow Bool}$ i $\mathtt{\lambda X.X}$ są abstrakcjami konstruktowymi, ale $\mathtt{\lambda x.x}$ jest abstrakcją na termach. 
Do konstruktora $\mathtt{Tb}$ możemy zaaplikować $\mathtt{Bool}$ i dostaniemy $\mathtt{(\lambda X.X \rightarrow Bool)Bool}$ równoważne $\mathtt{Bool \rightarrow Bool}$. Jak widać, użyliśmy słowa \textit{równoważne}. W rachunku lambda z typami prostymi sposób konstrukcji typów gwarantował nam, że dwa typy $\mathtt{T_1}$ i $\mathtt{T_2}$ na pewno są różne (zakładając, że typy bazowe były sobie różne). W $\lambda_{\omega}$ jest inaczej -- konstruktory tego systemu możemy podzielić na klasy równoważności. Do klasy $\mathtt{Bool \rightarrow Bool}$ należą również $\mathtt{ (Tb^n) Bool }$ dla $n$ naturalnego, a $\mathtt{T^n}$ oznacza aplikację $n$ konstruktorów $\mathtt{T}$.
Zauważmy, że odpowiednikiem takiej relacji równoważności w $\lambda_{\rightarrow}$  jest $\beta$-równoważność. W~świecie typów nazwiemy taką relację $\equiv$\footnote{formalnie zdefiniujemy tą relację w rozdziale 2.2.3}. Każdy konstruktor typu jest silnie normalizowalny i zachodzi własność Churcha-Rossera. Przez \texttt{nf(T)} oznaczamy postać normalną konstruktora rodzaju \texttt{T}.
Dodatkowo wprowadzimy następującą regułę:
$\mathtt{\frac{\Gamma \vdash t:S \hspace{1em} S\equiv T  }{\Gamma \vdash t:T}}$
mówiącą, że jeżeli \texttt{S} jest konstruktorem termu \texttt{t}, to dowolny konstruktor \texttt{S} równoważny z \texttt{T} również jest konstruktorem \texttt{t}.  

Niestety, w tak zdefiniowanym systemie powstaje jeden problem. Nie chcielibyśmy, aby $\mathtt{Bool \hspace{0.2em} Bool}$ było dozwolone, tak samo, jak w świecie termów nie chcieliśmy, by $\mathtt{true \hspace{0.2em} true}$ było dozwolone. W świecie termów, by rozwiązać ten problem, wprowadziliśmy typy na termach, w świecie typów wprowadzimy \textit{rodzaje} na konstruktorach. Piszemy, że $\mathtt{T::K}$, czyli konstruktor $\mathtt{T}$ jest rodzaju $\mathtt{K}$. Wprowadzimy też jeden rodzaj bazowy $\mathtt{\ast}$.

Wszystkie typy, jakie pojawiły się w $\lambda_{\rightarrow}$, są rodzaju $\mathtt{\ast}$. Np. $\mathtt{Bool :: \ast}$, $\mathtt{Nat \rightarrow Nat}$, $\mathtt{(Bool \rightarrow Nat) \rightarrow Nat} :: \ast$, itd. Rodzaj $\mathtt{\ast \Rightarrow \ast}$ będzie odpowiadał funkcjom z konstruktorów w konstruktory, np. $\mathtt{\lambda X.X \rightarrow Bool::\ast \Rightarrow \ast }$. $\mathtt{\ast \Rightarrow \ast \Rightarrow \ast}$ bierze konstruktor i zwraca funkcję konstruktorową, np. $\mathtt{\lambda X. \lambda Y. X \rightarrow Y::\ast \Rightarrow \ast \Rightarrow \ast}$, itd. 

Teraz możemy $\lambda_{\rightarrow}$ rozszerzyć o następujące konstrukcje: 
\begin{itemize}
    \item rodzaje \\ \\
\begin{tabular}{| l c r |}
  \hline
  $\mathtt{K}$ ::= &  & \textit{rodzaje}  \\
   & $\mathtt{\ast}$ & \textit{rodzaj wszystkich typów} \\
   & $\mathtt{K \Rightarrow K}$ & \textit{rodzaj funkcji typowej} \\
  \hline
\end{tabular}
    \item abstrakcję i aplikację typową na typach \\ \\
\begin{tabular}{| l c r |}
   \hline
   $\mathtt{T}$ ::= & $\dots$ & \textit{typy}  \\
   & $\mathtt{\lambda X::K.T}$ & \textit{abstrakcja konstruktorowa anotowana} \\
   & $\mathtt{\lambda X.T}$ & \textit{abstrakcja konstruktorowa nieanotowana} \\
   & $\mathtt{T\;T}$ & \textit{aplikacji konstruktorowa} \\
   \hline
\end{tabular}
\end{itemize} 

Powstaje pytanie, czy wszystkie konstruktory są typami? Otóż nie, typy to konstruktory rodzaju $\mathtt{\ast}$.

\subsubsection{System $F$}

System F jest systemem, w którym dodatkowo, oprócz termów zależnych od termów, mamy termy zależne od typów. Wprowadzimy trzeci już rodzaj abstrakcji i aplikacji, poprzedni był w świecie typów, ten będzie w świecie termów. Znana jest nam funkcja identycznościowa $\mathtt{\lambda x.x}$, w $\lambda_{\rightarrow}$ możemy ją napisać na wiele sposób: $\mathtt{\lambda x:Bool.x}$, $\mathtt{\lambda x:Nat.x}$, $\mathtt{\lambda x:Bool \rightarrow Nat.x}$. W systemie $F$ możemy wszystkie te funkcje zapisać jako: $\mathtt{\lambda X. \lambda x:X.x}$. Zauważmy, że ten \textit{term} przyjmuje jako pierwszy argument typ, następnie term tego typu i zwraca term. Przykładem użycia takiego termu mogą być: $(\mathtt{\lambda X. \lambda x:X.x)\;[Bool]\;true}$, co daje $\mathtt{true}$, albo $(\mathtt{\lambda X. \lambda x:X.x)\;[Nat]\;1}$, co daje $\mathtt{1}$. W ten sposób powstała nam \textit{uniwersalna} funkcja identycznościowa, której nadamy tzw. uniwersalny typ: $\mathtt{\lambda X. \lambda x:X.x \; : \; \forall X.X \rightarrow X}$. Dodatkowo, jako że dodaliśmy już do systemu rodzaje, napiszemy $\mathtt{\lambda X::\ast. \lambda x:X.x \; : \; \forall X::\ast.X \rightarrow X \; ::\ast}$. 

Czy moglibyśmy napisać $\mathtt{\lambda X::\ast \Rightarrow \ast. \lambda x:X.x \; : \; \forall X::\ast \Rightarrow \ast.X \rightarrow X \; ::\ast \Rightarrow \ast}\;$? Jak już mówiliśmy, tylko konstruktory rodzaju $\mathtt{\ast}$ są typami, więc powyższy term nie jest dobry.

Po tym krótkim wstępie możemy już zdefiniować odziedziczone z systemu $F$ własności takie, jak:  
\begin{itemize}
    \item abstrakcję i aplikację typową na termach \\ \\
\begin{tabular}{| l c r |}
  \hline
  $\mathtt{t}$ ::= & $\dots$ & \textit{termy} \\
   & $\mathtt{\lambda X::K.t}$ & \textit{abstrakcja typowa anotowana} \\
   & $\mathtt{\lambda X::K.t}$ & \textit{abstrakcja typowa nieanotowana}  \\
   & $\mathtt{t[T]}$ & \textit{aplikacja typowa} \\
  \hline
\end{tabular}

    \item typ uniwersalny \\ \\
\begin{tabular}{| l c r |}
  \hline
  $\mathtt{T}$ ::= & $\dots$ & \textit{typy}  \\
   & $\mathtt{\forall X::K.T}$ & \textit{typ uniwersalny anotowany} \\
   & $\mathtt{\forall X.T}$ & \textit{typ uniwersalny nieanotowany} \\
  \hline
\end{tabular}

\end{itemize}


\subsection{Typowanie}

\subsubsection{Kontekst}

\emph{Kontekst typowania} opisany jest następującą składnią abstrakcyjną: \\

\begin{tabular}{ | l c r | }
	\hline
	$\mathtt{\Gamma}$ ::= & & \textit{kontekst}\\
	& $\mathtt{\emptyset}$ & \textit{pusty kontekst} \\
	& $\mathtt{\Gamma,x:T}$& \textit{wiązanie typu} \\ 
	& $\mathtt{\Gamma,X::K}$& \textit{wiązanie rodzaju} \\ 
	\hline
\end{tabular} \\
	
Konteksty typowania bedziemy często traktować jako skończone zbiory wiązań i będziemy
używać teoriomnogościowych symboli na nich. Np. przynależność do kontekstu formalnie
definiujemy jako:
\[
	\frac{}{B \in \Gamma,B}
	\qquad
	\frac{B \in \Gamma}{B \in \Gamma,B'}
\]
Definicje pozostałych operacji teoriomnogościowych są na tyle naturalne, że zostawiamy
je Czytelnikowi do uzupełnienia.


\subsubsection{Podstawienia}
Oprócz zwykłego podstawienia za zmienne, które pozostawiamy Czytelnikowi do uzupełnienia, powinniśmy zdefiniować podstawienie za zmienne konstruktorowe.

    \begin{itemize}
        \item $\mathtt{[Y \mapsto T]X = \begin{cases}  \texttt{T} & Y=X \\ \texttt{X} & \text{w.p.p} \end{cases}}$
        \item $\mathtt{[Y \mapsto T](X_1 \; X_2)  = [Y \mapsto T]X_1 [Y \mapsto T] X_2 }$
        \item $\mathtt{[Y \mapsto T](S_1 \rightarrow S_2) = [Y \mapsto T]S_1 \rightarrow [Y \mapsto T]S_2} $
        \item $\mathtt{[Y \mapsto T]\forall X.S}  = \begin{cases} \mathtt{\forall X.S} & Y=X \text{ lub } Y \notin FV(S) \\ \mathtt{\forall X.[Y \mapsto T]S} & X \notin FV(S)  \text{ i } Y \in FV(S) \end{cases}$
        \item $\mathtt{[Y := T]\lambda X.S }= \begin{cases} \mathtt{\lambda X.S} & Y=X \text{ lub } Y \notin FV(S)\\ \mathtt{\lambda X.[Y \mapsto T]S}  & X \notin FV(S)  \text{ i } Y \in FV(S) \end{cases}$
    \end{itemize}

\subsubsection{Relacja $\equiv$}

Jak wspomnieliśmy w rozdziale 2.1.1, definiujemy na typach relację równoważności. W poniższych wzorach $\mathtt{S,S_1,S_2,T,T_1,T_2}$ to typy, \texttt{K} to rodzaj. Następujące trzy reguły:
	\[\mathtt{
		\frac{}{T \equiv T}
                \qquad
                \frac{S \equiv T}{T \equiv S}}
                \qquad
                \frac{S \equiv U \;\; U \equiv T}{S \equiv T}
	\]
gwarantują nam równoważność relacji $\equiv$. Pozostałe reguły jak następuje:
	\[\mathtt{
		\frac{S_1 \equiv T_1 \;\; S_2 \equiv T_2}{S_1 \rightarrow S_2 \equiv T_1 \rightarrow T_2}
                \qquad
                \frac{S_1 \equiv T_1 \;\; S_2 \equiv T_2}{S_1\;S_2 \equiv T_1\;T_2}}
	\]
	\[\mathtt{
		\frac{S \equiv T}{\lambda X::K.S \equiv \lambda X::K.T}
                \qquad
                (\lambda X::K.S)T \equiv [X \mapsto T]S}
	\] 

definiują równoważność funkcji typowych, aplikacji i abstrakcji konstruktorowych oraz typów uniwersalnych.

\subsubsection{Reguły znajdowania rodzaju}

W systemie $F_\omega$ każdemu poprawnie zbudowanemu typowi przyporządkowujemy rodzaj.
Przyporządkowanie to określa relacja $(.\vdash.::.)$ zdefiniowana następująco.

Jeżeli zachodzi $\mathtt{\Gamma \vdash T :: K}$, to powiemy, że
\emph {typ $\mathtt{T}$ jest rodzaju $\mathtt{K}$ w~kontekście $\mathtt{\Gamma}$}, gdzie
relacja określenia rodzaju $(.\vdash.::.) \subseteq \Gamma \times T \times K$
jest najmniejszą relacją zamkniętą na reguły:
	\[ \mathtt{
		\frac{X::K \in \Gamma}{\Gamma \vdash X::K}
		\qquad
		\frac{\Gamma \vdash T_1 :: K_{1} \Rightarrow K_{2} \hspace{1em} \Gamma \vdash T_2 :: K_{1}}{\Gamma \vdash T_1 T_2 :: K_{2}}}
	\]
	\[ \mathtt{
		\frac{\Gamma \vdash X :: K_1 \hspace{1em} \Gamma \vdash T :: K_2}{\Gamma \vdash \lambda X::K_1. T :: K_{1} \Rightarrow K_{2}}
		\qquad
		\frac{\Gamma \vdash X :: K \hspace{1em} \Gamma \vdash T :: \ast}{\Gamma \vdash \forall X::K. T :: \ast}}
	\]
	\[\mathtt{
		\frac{\Gamma \vdash T_1 : \ast \hspace{1em} \Gamma \vdash T_2 : \ast}{\Gamma \vdash T_1 \rightarrow T_2 : \ast}}
	\]	


\subsubsection{Reguły typowania}

Jesteśmy już gotowi przedstawić reguły typowania zdefiniowanego wyżej systemu $F_\omega$.
Każdemu poprawnie zbudowanemu termowi przyporządkowujemy typ. Przyporządkowanie to określa relacja $(.\vdash.:.)$ zdefiniowana następująco.

	\[\mathtt{
		\frac{x:T \in \Gamma}{\Gamma \vdash x:T}
                \qquad
                \frac{\Gamma \vdash T_1::\ast \hspace{1.5em} \Gamma,x:T_1 \vdash t_2:T_2}{\Gamma \vdash \lambda x:T_1.t_2\;:\;T_1 \rightarrow T_2}}
	\]
	\[\mathtt{
		\frac{\Gamma \vdash t_1:T_1 \rightarrow T_2 \hspace{1.5em} \Gamma \vdash t_2:T_1}{ \Gamma \vdash t_1 \; t_2 : T_2}
                \qquad
                \frac{\Gamma \vdash t:S \hspace{1.5em} S \equiv T \hspace{1.5em} \Gamma \vdash T::\ast}{\Gamma \vdash t:T}}
	\]
	\[\mathtt{
		\frac{\Gamma,X::K \vdash t:T }{\Gamma \vdash \lambda X::K.t \;:\forall X::K.T}
                \qquad
                \frac{\Gamma \vdash t:\forall X::K.T \hspace{1.5em} \Gamma \vdash T'::K}{\Gamma \vdash t[T']\;:[X \mapsto T']T}}
	\]


\subsection{Ewaluacja}

Wartości w $F_\omega$ zdefiniujemy dokładnie jak w $\lambda_{\rightarrow}$. \\ \\
\begin{tabular}{| l c r |}
  \hline
  $\mathtt{v}$ ::= &  & \textit{wartości}  \\
   & $\mathtt{\lambda x:T.t}$ & \textit{wartość abstrakcji} \\
  \hline
\end{tabular} \\

Ewaluacja przebiega w sposób standardowy dla aplikacji i abstrakcji termów. Teraz, dla czytelności, przetoczymy te reguły ewaluacji ($\mathtt{t_1,t_1',t_2,t_2',t}$ to termy, \texttt{v} to wartość, \texttt{x:T} to zmienna \texttt{x} typu \texttt{T}):
	\[\mathtt{
		\frac{t_1 \longrightarrow t_1'}{t_1\;t_2 \longrightarrow t_1'\;t_2}
                \qquad
                \frac{t_2 \longrightarrow t_2'}{v_1\;t_2 \longrightarrow v_1\;t_2'}}
	\]
 	\[\mathtt{
		(\lambda x:T.t)v \longrightarrow [x \mapsto v]t}
	\]
Do tego dochodzą reguły dla nowych w języku abstrakcji typowych i aplikacji typowych.
	\[\mathtt{
		\frac{t \longrightarrow t'}{t[T] \longrightarrow t'[T]}}
	\]
 	\[\mathtt{
		(\lambda X::K.t)[T] \longrightarrow [X \mapsto T]t}
	\]
%%%%%%%%%%%%%%%%%%%%%%%%%%%%%%%%%%%%%%%%%%%%%%%%%%%%%%%%%%%%%%%%%%%%%%%%%%%%%%%%
%%%%%%%%%%%%%%%%%%%%%%%%%%%%%%%%%%%%%%%%%%%%%%%%%%%%%%%%%%%%%%%%%%%%%%%%%%%%%%%%
\section{Rozszerzenia $F_\omega$}
%%%%%%%%%%%%%%%%%%%%%%%%%%%%%%%%%%%%%%%%%%%%%%%%%%%%%%%%%%%%%%%%%%%%%%%%%%%%%%%%
%%%%%%%%%%%%%%%%%%%%%%%%%%%%%%%%%%%%%%%%%%%%%%%%%%%%%%%%%%%%%%%%%%%%%%%%%%%%%%%%
\setcounter{equation}{0}

W rozdziale tym chcielibyśmy poruszyć, jak w systemie $F_\omega$ zdefiniować najprostsze
konstrukcje, takie jak wyrażenia arytmetyczne i logiczne, warianty, sekwencje wyrażeń, 
typy egzystencjalne, rekordy i inne. Pokażemy również, jak przebiega typowanie, ewaluacja i 
gdzieniegdzie dodamy reguły tworzenia rodzaju.

Chcielibysmy podkreślić, że następująca reguła typowania:

 	\[\mathtt{ \frac{\Gamma \vdash t:T \;\;\; S\equiv T \;\;\; \Gamma \vdash S::\ast}{\Gamma \vdash t:S}
		}
	\]
bardzo ułatwia definiowanie reguł typowania w $F_\omega$. W większości przypadków są one takie same lub
lekko zmodyfikowane, dlatego nie powinny nastręczać trudności.

\subsection{wyrażenia arytmetyczne i logiczne}

Wyreżania arytmetyczne i logiczne to część, bez której żaden język się nie obędzie. Oczywiście można je sobie zakodować w systemie $F_\omega$, ale normą są wbudowane w język wyrażenia. Termy, typy i wartości wyrażeń zdefiniujemy następująco: \\

\begin{tabular}{| l c r |}
  \hline
  $\mathtt{t}$ ::= & $\dots$ & \textit{termy}  \\
   & $\mathtt{true}$ & \textit{prawda}  \\
   & $\mathtt{false}$ & \textit{fałsz} \\
   & $\mathtt{zero}$ & \textit{zero} \\
   & $\mathtt{succ\;t}$ & \textit{następnik}  \\
   & $\mathtt{pred\;t}$ & \textit{poprzednik} \\
   & $\mathtt{iszero}$ & \textit{test na zero} \\
   & $\mathtt{if\;t\;then\;t\;else\;t}$ & \textit{warunek} \\
   & & \\
  $\mathtt{T}$ ::= & $\dots$ & \textit{typy} \\
   & $\mathtt{Nat}$ & \textit{typ liczbowy} \\
   & $\mathtt{Bool}$ & \textit{typ boolowski} \\
   & & \\
  $\mathtt{v}$ ::= & $\dots$ & \textit{typy} \\
   & $\mathtt{true}$ & \textit{wartość prawdy} \\
   & $\mathtt{false}$ & \textit{wartość fałszu} \\
   & $\mathtt{nv}$ & \textit{wartość liczbowa} \\
   & & \\
  $\mathtt{nv}$ ::= & $\dots$ & \textit{wartość liczbowa} \\
   & $\mathtt{zero}$ & \textit{wartość zera} \\
   & $\mathtt{succ\;nv}$ & \textit{wartość następnika} \\
  \hline
\end{tabular} \\

Na pewno musimy dodać reguły tworzenia rodzaju dla \texttt{Nat} i \texttt{Bool}, którym nadamy rodzaj~$\mathtt{\ast}$:
 	\[\mathtt{ \frac{}{\Gamma \vdash Bool::\ast} \qquad \frac{}{\Gamma \vdash Nat::\ast}
		}
	\]

Typowanie wygląda dokładnie tak samo jak w rachunku lambda z typami prostymi. Możemy sobie pozwolić na takie reguły dzięki regule XXX.
Przykładowo, nie tylko termy typu \texttt{Nat} mogą się dobrze otypować, gdy zaaplikujemy je do \texttt{succ}. Dla $\mathtt{t:(\lambda X.X)Nat}$
otrzymamy:

 	\[\mathtt{ \frac{\frac{\Gamma \vdash t:(\lambda X.X)Nat\hspace{2em}\Gamma \vdash (\lambda X.X)Nat \equiv Nat \hspace{2em}\Gamma \vdash (\lambda X.X)Nat::\ast}{\Gamma \vdash t:Nat}}{\Gamma \vdash succ\;t:Nat} 
		}
	\]
Nie musimy również pisać reguł typu:~\[\mathtt{\frac{\Gamma \vdash t_1:Bool \;\;\; \Gamma \vdash t_2:T \;\;\; \Gamma \vdash t_3:T\;\;\; \Gamma \vdash T::\ast}{\Gamma \vdash if\;t_1\;then\;t_2\;else\;t_3:T} 
		}
	\]
ponieważ posiadanie typu przez term $\mathtt{t_2}$ gwarantuje nam, że ten typ będzie rodzaju $\mathtt{\ast}$.
Stąd, reguły typowania wyrażeń arytmetycznych i logicznych wyglądają następująco w $F_\omega$:~\[\mathtt{ \frac{}{\Gamma \vdash true:Bool} \qquad \frac{}{\Gamma \vdash false:Bool} \quad \frac{\Gamma \vdash t:Nat}{\Gamma \vdash iszero\;t:Bool}
		}
	\]
 	\[\mathtt{\frac{\Gamma \vdash t_1:Bool \;\;\; \Gamma \vdash t_2:T \;\;\; \Gamma \vdash t_3:T}{\Gamma \vdash if\;t_1\;then\;t_2\;else\;t_3:T} 
		}
	\]
 	\[\mathtt{ \frac{}{\Gamma \vdash zero:Nat} \qquad \frac{\Gamma \vdash t:Nat}{\Gamma \vdash succ\;t:Nat} \qquad \frac{\Gamma \vdash t:Nat}{\Gamma \vdash pred\;t:Nat} 
		}
	\]
Zdefiniowanie reguł ewaluacji pozostawiamy Czytelnikowi.

\subsection{Unit i sekwencje}

W rachunku lambda z typami prostymi dodaliśmy do składnię języka rozszerzaliśmy o konstrukcje takie, jak: 

\begin{tabular}{| l c r |}
  \hline
  $\mathtt{t}$ ::= & $\dots$ & \textit{termy}  \\
   & $\mathtt{unit}$ & \textit{term unit} \\
  $\mathtt{T}$ ::= & $\dots$ & \textit{typy}  \\
   & $\mathtt{Unit}$ & \textit{typ unit} \\
  $\mathtt{v}$ ::= & $\dots$ & \textit{wartości}  \\
   & $\mathtt{unit}$ & \textit{wartość unit} \\
  \hline
\end{tabular} \\

natomiast typowanie przebiegało następująco:~\[\mathtt{ \frac{}{\Gamma \vdash unit:Unit}
		}
	\]
a sekwencje definiowaliśmy jako:~\[\mathtt{ t_1;t_2 \;\overset{def}{=}\; (\lambda x:Unit.t_2)t_1 \hspace{3em} \text{gdzie x $\notin$ FV($t_2$)}
		}
	\]
Aby pozostać przy wbudowanym unit w język wystarczy dodać regułę znajdowania rodzaju dla typu \texttt{Unit}:~\[\mathtt{ \frac{}{\Gamma \vdash Unit::\ast}
		}
	\]

W rachunku $F_\omega$ pojawia się możliwość zakodowania \texttt{unit} i \texttt{Unit}. Robimy to w taki sposób:
$\mathtt{unit \;\overset{def}{=}\;= \lambda X::\ast.\lambda x:X.x}$ \\
$\mathtt{Unit \;\overset{def}{=}\; \forall X::\ast.X \rightarrow X}$ \\


\subsection{Anotacje typowe}

Anotacje typowe są przydatną konstrukcją używaną na przykład przy typach egzystencjalnych.\\
\begin{tabular}{| l c r |}
  \hline
  $\mathtt{t}$ ::= & $\cdots$ & \textit{termy}  \\
   & $\mathtt{t\;as\;T}$ &\textit{anotacja typowa} \\
  \hline
\end{tabular} \\

Ewaluacja i typowanie nie zmieniają się.

\subsection{Definicje lokalne}

Do zbioru termów dodamy następujące dwie konstrukcje:\\

\begin{tabular}{| l c r |}
  \hline
  t ::= &  &  \\
   & $\textrm{let} \ x = t \ \textrm{in} \ t$ & \textit{zmienna lokalna} \\
   & $\textrm{tlet} \ X = T \ \textrm{in} \ t$ & \textit{typ lokalny} \\
  \hline
\end{tabular} \\

Oczywiście możemy zdefiniować \texttt{let} i \texttt{tlet} jako:
 	\[\mathtt{let\;x=t_1\;in\;t_2 \;\overset{def}{=}\;(\lambda x:T.t_2)t_1 \hspace{3em}\text{gdzie}\; t_1:T 
		}
	\]
 	\[\mathtt{tlet\;X=T\;in\;t \;\overset{def}{=}\;(\lambda X::K.t)T \hspace{3em}\text{gdzie}\; T::K 
		}
	\]
Reguły typowania i ewaluacji pozostają bez zmian dla \texttt{let}, ale zdefiniujemy je dla \texttt{tlet}.
Typowanie:
 	\[\mathtt{ \frac{\Gamma,X:S \vdash t:T}{\Gamma \vdash tlet\;X=S\;in\;t : [X \mapsto S]T}
		}
	\]
i ewaluacja:
 	\[\mathtt{ tlet\;X=S\;in\;t \longrightarrow [X \mapsto S]t
		}
	\]
 	\[\mathtt{ \frac{t \longrightarrow t'}{tlet\;X=T\;in\;t \longrightarrow tlet\;X=T\;in\;t'}
		}
	\]

\subsection{Pary}

Składnię par zdefiniujemy następująco:\\

\begin{tabular}{| l c r |}
  \hline
  $\mathtt{t}$ ::= & $\dots$ & \textit{termy}  \\
   & $\mathtt{(t,t)}$ & \textit{para} \\
   & $\mathtt{t.1}$ & \textit{projekcja na pierwszy element} \\ 
   & $\mathtt{t.2}$ & \textit{projekcja na drugi element} \\ 
   & & \\
  $\mathtt{T}$ ::= & $\dots$ & \textit{typy}  \\
   & $\mathtt{T_1\times T_2}$ & \textit{typ pary} \\
   & & \\
  $\mathtt{v}$ ::= & $\dots$ & \textit{wartości}  \\
   & $\mathtt{(v,v)}$ & \textit{wartość pary} \\
  \hline
\end{tabular} \\

Do relacji tworzenia rodzaju dodamy regułę nadającą rodzaj typowi $\mathtt{\{l_i:T_i^{\;\;\;i \in 1..n}\}}$:~\[\mathtt{
             \frac{\Gamma \vdash T_1::\ast \hspace{2em} \Gamma \vdash T_2::\ast}{\Gamma \vdash T_1\times T_2::\ast}
		}
	\]
oraz wprowadzimy niewielkie zmiany w regułach typowania:~\[\mathtt{
             \frac{\Gamma \vdash t_1:T_1 \hspace{2em} \Gamma \vdash t_2:T_2 \hspace{2em}\Gamma \vdash T_1\times T_2::\ast}{\Gamma \vdash (t_1,t_2):T_1\times T_2}
            }
	\]
 	\[\mathtt{
           \frac{\Gamma \vdash t : T_1\times T_2}{\Gamma \vdash t.1 : T_1} 
           \qquad
           \frac{\Gamma \vdash t : T_1\times T_2}{\Gamma \vdash t.2 : T_2} 
            }
	\]
a ewaluację pozostawimy bez zmian: \\
 	\[\mathtt{
                \frac{t \longrightarrow t'}{t.i \longrightarrow t'.i} \hspace{3em} \text{gdzie } i\in\{1,2\}
		}
	\]
 	\[\mathtt{
                \frac{t_1 \longrightarrow t_1'}{(t_1,t_2) \longrightarrow (t_1',t_2)} 
                \qquad
                \frac{t_2 \longrightarrow t_2'}{(v_1,t_2) \longrightarrow (v_1,t_2')} 
		}
	\]
 	\[\mathtt{
                (v_1,v_2).i \longrightarrow v_i \hspace{3em} \text{gdzie } i\in\{1,2\}
		}
	\]

\subsection{Rekordy}

Składnię rekordów zdefiniujemy następująco:\\

\begin{tabular}{| l c r |}
  \hline
  $\mathtt{t}$ ::= & $\dots$ & \textit{termy}  \\
   & $\mathtt{\{l_i=t_i^{\;\;\;i \in 1..n}\}}$ & \textit{rekord} \\
   & $\mathtt{t.l}$ & \textit{projekcja} \\ 
   & & \\
  $\mathtt{T}$ ::= & $\dots$ & \textit{typy}  \\
   & $\mathtt{\{l_i:T_i^{\;\;\;i \in 1..n}\}}$ & \textit{typ rekordu} \\
   & & \\
  $\mathtt{v}$ ::= & $\dots$ & \textit{wartości}  \\
   & $\mathtt{\{l_i=v_i^{\;\;\;i \in 1..n}\}}$ & \textit{wartość rekordu} \\
  \hline
\end{tabular} \\

Do relacji tworzenia rodzaju dodamy regułę nadającą rodzaj typowi $\mathtt{\{l_i:T_i^{\;\;\;i \in 1..n}\}}$:~\[\mathtt{
             \frac{\Gamma \vdash T_1::\ast \;\dots\;\Gamma \vdash T_n::\ast}{\Gamma \vdash \{l_i:T_i^{\;\;\;i \in 1..n}\}::\ast}
		}
	\]
oraz wprowadzimy niewielkie zmiany w regułach typowania:~\[\mathtt{
             \frac{\Gamma \vdash t_1:T_1 \;\dots\;\Gamma \vdash t_n:T_n \hspace{2em} \Gamma \vdash \{l_i:T_i^{\;\;\;i \in 1..n}\}::\ast}{\Gamma \vdash \{l_i=t_i^{\;\;\;i \in 1..n}\}:\{l_i:T_i^{\;\;\;i \in 1..n}\}}
            }
	\]
 	\[\mathtt{
           \frac{\Gamma \vdash t : \{l_i:T_i^{\;\;\;i \in 1..n}\}}{\Gamma \vdash t.i : T_i} 
            }
	\]
a ewaluację pozostawimy bez zmian: \\
 	\[\mathtt{
                \{l_i=v_i^{\;\;\;i \in 1..n}\}.i \longrightarrow v_i
                \qquad
                \frac{t \longrightarrow t'}{t.i \longrightarrow t'.i}
		}
	\]
\small{
 	\[\mathtt{
                \frac{t_i \longrightarrow t_i'}{\{l_1=v_1, \dots, l_{i-1}=v_{i-1}, l_i=t_i, \dots, l_n=t_n\} \longrightarrow \{l_1=v_1, \dots, l_{i-1}=v_{i-1}, l_i=t_i', \dots, l_n=t_n\}}
		}
	\]
}

\subsection{Sumy}

Składnię sum zdefiniujemy następująco:\\

\begin{tabular}{| l c r |}
  \hline
  $\mathtt{t}$ ::= & $\dots$ & \textit{termy}  \\
   & $\mathtt{inl\;t}$ & \textit{tagowanie lewe} \\
   & $\mathtt{inr\;t}$ & \textit{tagowanie prawe} \\
   & $\mathtt{case\;t\;of\;inl\;x\Rightarrow t\;|\;inr\;x\Rightarrow t }$ & \textit{case}\\
   & & \\
  $\mathtt{T}$ ::= & $\dots$ & \textit{typy}  \\
   & $\mathtt{T+T}$ & \textit{typ sumy}\\
  \hline
\end{tabular} \\

Typ sumy dostanie rodzaj $\mathtt{\ast}$:
 	\[\mathtt{
             \frac{\Gamma \vdash T_1::\ast \hspace{2em} \Gamma \vdash T_2::\ast}{\Gamma \vdash T_1+T_2::\ast}
		}
	\]
a w regułach typowania wprowadzimy małe zmiany:
 	\[\mathtt{ \frac{\Gamma \vdash t_0:T_1+T_2 \hspace{2em} \Gamma,x_1:T_1 \vdash t_1:T \hspace{2em} \Gamma,x_2:T_2 \vdash t_2:T}{\Gamma \vdash case\;t_0\;of\;inl\;x_1\Rightarrow t_1\;|\;inr\;x_2\Rightarrow t_2 : T}
		}
	\]
 	\[\mathtt{ \frac{\Gamma \vdash t_1 : T_1 \hspace{2em} \Gamma \vdash T_1+T_2::\ast}{\Gamma \vdash inl\;t_1:T_1+T_2}
		}
	\]
 	\[\mathtt{ \frac{\Gamma \vdash t_2 : T_2 \hspace{2em} \Gamma \vdash T_1+T_2::\ast}{\Gamma \vdash inr\;t_2:T_1+T_2}
		}
	\]
Ewaluację pozostawiamy bez zmian.


\subsection{Warianty}

Składnię wariantów zdefiniujemy następująco:\\

\begin{tabular}{| l c r |}
  \hline
  $\mathtt{t}$ ::= & $\dots$ & \textit{termy}  \\
   & $\mathtt{<l=t>\;as\;T}$ & \textit{tagowanie} \\
   & $\mathtt{case\;t\;of\;<l_i=x_i>\Rightarrow t_i^{\;\;\;i \in 1..n}}$ & \textit{case}\\
   & & \\
  $\mathtt{T}$ ::= & $\dots$ & \textit{typy}  \\
   & $\mathtt{<l_i:T_i^{\;\;\;i \in 1..n}>}$ & \textit{typ wariantu}\\
  \hline
\end{tabular} \\

Podobnie jak przy rekordach, typ wariantu dostanie rodzaj $\mathtt{\ast}$:
 	\[\mathtt{
             \frac{\Gamma \vdash T_1::\ast \;\dots\;\Gamma \vdash T_n::\ast}{\Gamma \vdash <l_i:T_i^{\;\;\;i \in 1..n}>::\ast}
		}
	\]
a w regułach typowania wprowadzimy małe zmiany:
 	\[\mathtt{ \frac{\Gamma \vdash t_0:<l_i:T_i^{\;\;\;i \in 1..n}> \hspace{2em} \Gamma,x_1:T_1 \vdash t_1:T \;\dots\; \Gamma,x_n:T_n \vdash t_n:T}{\Gamma \vdash case\;t_0\;of\;<l_i=x_i>\Rightarrow t_i^{\;\;\;i \in 1..n}\; : T}
		}
	\]
 	\[\mathtt{ \frac{\Gamma \vdash t_j : T_j \hspace{2em} \Gamma \vdash <l_i:T_i^{\;\;\;i \in 1..n}>::\ast}{\Gamma \vdash <l_j=t_j> as <l_i:T_i^{\;\;\;i \in 1..n}> \;: <l_i:\;T_i^{\;\;\;i \in 1..n}>}
		}
	\]
natomiast ewaluacja pozostanie bez zmian:
 	\[\mathtt{ case\;(<l_j=t_j>\;as\;T)\;of\;<l_i=x_i>\Rightarrow t_i^{\;\;\;i \in 1..n} \longrightarrow [x_j \mapsto v_j]t_j
		}
	\]
 	\[\mathtt{ \frac{t \longrightarrow t'}{case\;t\;of\;<l_i=x_i>\Rightarrow t_i^{\;\;\;i \in 1..n} \longrightarrow case\;t'\;of\;<l_i=x_i>\Rightarrow t_i^{\;\;\;i \in 1..n} }
		}
	\]
 	\[\mathtt{ \frac{t_j \longrightarrow t_j'}{<l_j=t_j>\;as\;T \longrightarrow <l_j=t_j'>\;as\;T }
		}
	\]

\subsection{Punkt stały}

Do składni języka dodamy jedynie jeden term:\\
\begin{tabular}{| l c r |}
  \hline
  $\mathtt{t}$ ::= & $\dots$ & \textit{termy}  \\
   & $\mathtt{fix \;x:T.t}$ & \textit{punkt stały}\\
  \hline
\end{tabular} \\

Typowanie zostanie lekko zmodyfikowane:\\
 	\[\mathtt{\frac{\Gamma, x:T \vdash t:S \hspace{2em} \Gamma \vdash T::\ast \hspace{2em} S\equiv T}{\Gamma \vdash fix\;x:T.t:T}}
	\]
a ewaluacja pozostanie bez zmian: \\
 	\[\mathtt{ fix\;x:T.t \longrightarrow [x \mapsto fix\;x:T.t]t
		}
	\]

\subsection{Listy}

Jako przykład wbudowanych typów danych wybraliśmy listy. Podobne rekursywne struktury, jak na przykład drzewa, możemy dodać do języka w analogiczny sposób, jednak rekurencyjne typy danych odwiodą nas od tej konieczności. \\
  
\begin{tabular}{| l c r |}
  \hline
  $\mathtt{t}$ ::= & $\dots$ & \textit{termy}  \\
   & $\mathtt{nil[T]}$ & \textit{lista pusta}  \\
   & $\mathtt{cons[T]\;t\;t}$ & \textit{konstruktor listy} \\
   & $\mathtt{isnil[T]\;t}$ & \textit{test na pustość listy} \\
   & $\mathtt{head[T]\;t\;t}$ & \textit{głowa listy} \\
   & $\mathtt{tail[T]\;t}$ & \textit{ogon listy} \\
   & & \\
  $\mathtt{T}$ ::= &  & \textit{typy} \\
   & $\mathtt{List\;T}$ & \textit{typ listy} \\
   & & \\
  $\mathtt{v}$ ::= &  & \textit{typy} \\
   & $\mathtt{nil\;[T]}$ & \textit{wartość pustej listy} \\
   & $\mathtt{cons\;[T]\;v\;v}$ & \textit{wartość listy niepustej} \\
  \hline
\end{tabular} \\

Do reguł rodzajowania dodamy regułę:
 	\[\mathtt{ \frac{\Gamma \vdash T::K}{\Gamma \vdash List\;T::K}
		}
	\]
czyli $\mathtt{List\;T}$ ma ten sam rodzaj, co jego elementy.
Reguły typowania nieznacznie się zmienią:
 	\[\mathtt{ \frac{\Gamma \vdash List\;T :: \ast}{\Gamma \vdash nil[T] : List\;T } \qquad \frac{\Gamma \vdash t_1:T \;\;\;\Gamma \vdash t_2:List\;T}{\Gamma \vdash List[T]\;t_1\;t_2:List\;T}
		}
	\]
 	\[\mathtt{ \frac{\Gamma \vdash t:List\;T }{\Gamma \vdash head[T]\;t : T } \qquad \frac{\Gamma \vdash t:List\;T }{\Gamma \vdash tail[T]\;t : List\;T }
		}
	\]
a ewaluacja pozostanie ta sama, uzupełnienie jej pozostawiamy Czytelnikowi.
\subsection{Typy egzystencjalne}

System $F_\omega$ jest już w stanie zakodować typy egzystencjalne, choć wbudowane typy egzystencjalne niczemu nie szkodzą. Pokażemy oba podejścia do tego problemu, zaczynając od przedstawienia składni: \\

\begin{tabular}{| l c r |}
  \hline
  $\mathtt{t}$ ::= & $\cdots$ & \textit{termy}  \\
   & $\mathtt{\{^*T::K,t\}\;as\;T}$ & \textit{pakowanie} \\
   & $\mathtt{let\;\{X,x\}=t\;in\;t}$ & \textit{odpakowanie} \\
   & & \\
  $\mathtt{T}$ ::= & $\cdots$ & \textit{typy}  \\
   & $\mathtt{\{\exists X::K,T\}}$ & \textit{typ egzystencjalny} \\
   & & \\
  $\mathtt{v}$ ::= & $\cdots$ & \textit{wartości}  \\
   & $\mathtt{\{^*T,v\}\;as\;T}$ & \textit{pakowanie} \\
  \hline
\end{tabular} \\

W systemie $F_\omega$ powyższe elementy języka możemy zdefiniować następująco: \\
$\mathtt{\{\exists X::K,T\}\;\overset{def}{=}\;\forall Y::\ast.(\forall X::K.T \rightarrow Y) \rightarrow Y}$ \\
$\mathtt{\{^*U::K,t\}\;as\; \{\exists X::K,T\}\; \overset{def}{=} \;let\;x=t\;in\;\lambda Y::\ast.(\lambda f:\forall X::K.T \rightarrow Y) .f\;[U]\;x}$ \\
$\mathtt{let\;\{X::K,x\}=t\;in\;t'\;\overset{def}{=}\;t[T'](\lambda X::K.\lambda x:T.t')\hspace{3em} \text{gdzie $\mathtt{t':T'}$} }$ \\ \\

Pokażemy, że zachodzą podstawowe własności pakowania i odpakowania. \\
Rozważmy term $\mathtt{ \{^*U::K,t\}\;as\; \{\exists X::K,T\}}$. \\ 
$ \mathtt{ \{^*U::K,t\}\;as\; \{\exists X::K,T\} }$ \\ 
$ \mathtt{ = let\;x=t\;in\;\lambda Y::\ast.\lambda f:(\forall X::K.T \rightarrow Y).f[U]x = }$ \\
$ \mathtt{ = (\lambda x:[X \mapsto U]T.\lambda Y::\ast.\lambda f:(\forall X::K.T \rightarrow Y).f[U]x)t = }$ \\
$ \mathtt{ \overset{t:[X \mapsto U]T}{=} \lambda Y::\ast.\lambda f:(\forall X::K.T \rightarrow Y).f[U](t:[X \mapsto U]T)}$ \\
co jest typu $\mathtt{\forall Y::\ast.(\forall X::K.T \rightarrow Y) \rightarrow Y}$, czyli z definicji $\mathtt{\{\exists X::K,T\}}$. \\

Rozważmy bardziej życiowy przykład, aby Czytelnik mógł jeszcze raz przeanalizować pakowanie. Oto typowanie w systemie $F$ przykładowego termu:\\ 
 	\[\mathtt{ \frac{\Gamma \vdash \{a=zero, f:\lambda x:Nat.succ\;x\}:[X \mapsto Nat]\{a:X,f:X \rightarrow Nat\}}{\Gamma \{^*X,\{a=zero, f:\lambda x:Nat.succ\;x\}\}\;as\;\{\exists X,\{a:X,f:X \rightarrow Nat\}\}}
		}
	\]
Następnie wyprowadzimy ten term w $F_\omega$: \\ \\
\small{
$ \mathtt{ \{Nat::K,\{a=zero, f:\lambda x:Nat.succ\;x\}\}\;as\; \{\exists X::K,\{a:X, f:X \rightarrow Nat\}\}= }$ \\ 
$ \mathtt{ = let\;x=\{a=zero, f:\lambda x:Nat.succ\;x\}\;in\;\lambda Y::\ast.\lambda f:(\forall X::K.\{a:X, f:X \rightarrow Nat\} \rightarrow Y).f[Nat]x = }$ \\
$ \mathtt{ = (\lambda x:[X \mapsto Nat]\{a:X, f:X \rightarrow Nat\}.\lambda Y::\ast.\lambda f:(\forall X::K.\{a:X, f:X \rightarrow Nat\} \rightarrow Y).f[Nat]x)}$\\
$ \mathtt{ \{a=zero, f:\lambda x:Nat.succ\;x\} \overset{\{a=zero, f:\lambda x:Nat.succ\;x\}:[X \mapsto Nat]\{a:X, f:X \rightarrow Nat\}}{=}  }$ \\
$ \mathtt{ = \lambda Y::\ast.\lambda f:(\forall X::K.\{a:X, f:X \rightarrow Nat\} \rightarrow Y).f[Nat]\{a=zero, f:\lambda x:Nat.succ\;x\}}$ \\ \\
co jest typu $\mathtt{\forall Y::\ast.(\forall X::K.\{a:X, f:X \rightarrow Nat\} \rightarrow Y) \rightarrow Y}$, czyli z definicji $\mathtt{\{\exists X::K,\{a:X, f:X \rightarrow Nat\}\}}$. \\
}

\normalsize{Uważne odpakowanie otrzymanego termu pozostawiamy Czytelnikowi jako ćwiczenie, my pozwolimy sobie przeprowadzać schemat wywodu:}\\
\scriptsize{
$ \mathtt{let\;\{X,x\}=\lambda Y::\ast.\lambda f:(\forall X::K.\{a:X, f:X \rightarrow Nat\} \rightarrow Y).f[Nat]\{a=zero, f:\lambda x:Nat.succ\;x\}\;in\;(x.f\;x.a)=}$ \\
$ \mathtt{=(\lambda Y::\ast.\lambda f:(\forall X::K.\{a:X, f:X \rightarrow Nat\} \rightarrow Y).f[Nat]\{a=zero, f:\lambda x:Nat.succ\;x\})[T'](\lambda X::K.\lambda x:T.(x.f\;x.a))=}$\\
}
$ \mathtt{=((\lambda X::K.\lambda x:T.(x.f\;x.a))[Nat](\{a=zero, f:\lambda x:Nat.succ\;x\}:[X \mapsto Nat]\{a:X, f:X \rightarrow Nat\}))=}$ \\
$ \mathtt{=(\{a=zero, f:\lambda x:Nat.succ\;x\}.f\;\{a=zero, f:\lambda x:Nat.succ\;x\}.a)=}$ \\
$ \mathtt{=(\lambda x:Nat.succ\;x)zero=succ\;zero}$ \\ 


\normalsize{
Przykłady powyższe obrazują działanie zakodowanych typów rekurencyjnych. Teraz zdefiniujemy wbudowane w język konstrukcje typów rekurencyjnych dla systemu $F_\omega$. Do definicji termów, typów i wartości dodaliśmy już elementy w tabelce na początku rozdziału. Pokażemy, w jaki sposób przebiega typowanie i ewaluacja. \\
}


Na pewno musimy określić, jaki rodzaj dostaje $\mathtt{\{\exists X::K,T\}}$:
 	\[\mathtt{ \frac{\Gamma,X::K \vdash T::\ast}{\Gamma \vdash \{\exists X::K,T\}::\ast}
		}
	\]
oraz zdefiniować relację znajdowania typu odpowiednią dla $F_\omega$:
 	\[\mathtt{ \frac{\Gamma \vdash t:[X \mapsto U]T\hspace{2em}\Gamma \vdash U::K \hspace{2em}\Gamma \vdash \{\exists X::K,T\}::\ast}{\Gamma \vdash \{^*U::K,t\}\;as\;\{\exists X::K,T\} \; : \; \{\exists X::K,T\}}
		}
	\]
 	\[\mathtt{ \frac{\Gamma \vdash t_1:\{\exists X::K,T_1\}\hspace{2em}\Gamma,X::K,x:T_1 \vdash t_2:T_2}{\Gamma \vdash let\;\{X,x\}=t_1\;in\;t_2:T_2}
		}
	\]
Ewaluacja nieznacznie się zmieni:
 	\[\mathtt{ let\;\{X,x\}=(\{^*U::K,v\}\;as\;T)\;\;in\;t \longrightarrow [X \mapsto U][x \mapsto v]t
		}
	\]
 	\[\mathtt{ \frac{t \longrightarrow t'}{\{^*U::K,t\}\;as\;T \longrightarrow \{^*U::K,t'\}\;as\;T}
		}
	\]
 	\[\mathtt{ \frac{t_1\longrightarrow t_1'}{let\;\{X,x\}=t_1\;\;in\;t_2 \longrightarrow let\;\{X,x\}=t_1'\;\;in\;t_2}
		}
	\]

\subsection{Uzupełnianie relacji $\mathtt{\equiv}$}

Po wprowadzeniu rozszerzeń z rozdziału trzeciego możemy rozszerzyć relację $\mathtt{\equiv}$:
        \[\mathtt{ \frac{S \equiv T}{\{\exists X::K,S\} \equiv \{\exists X::K,T\}}
                   \qquad
                   \frac{S \equiv T}{List\;S \equiv List\;T}
		}
	\]
        \[\mathtt{ \frac{T_1 \equiv T_1' \dots T_n \equiv T_n'}{\{l_i:T_i^{\;\;\;i \in 1..n}\} \equiv \{l_i:T_i'^{\;\;\;i \in 1..n}\}}
                   \qquad
                   \frac{T_1 \equiv T_1' \dots T_n \equiv T_n'}{<l_i:T_i^{\;\;\;i \in 1..n}> \equiv <l_i:T_i'^{\;\;\;i \in 1..n}>}
		}
	\]
%%%%%%%%%%%%%%%%%%%%%%%%%%%%%%%%%%%%%%%%%%%%%%%%%%%%%%%%%%%%%%%%%%%%%%%%%%%%%%%%
%%%%%%%%%%%%%%%%%%%%%%%%%%%%%%%%%%%%%%%%%%%%%%%%%%%%%%%%%%%%%%%%%%%%%%%%%%%%%%%%
\section{Rekonstrukcja typów}
%%%%%%%%%%%%%%%%%%%%%%%%%%%%%%%%%%%%%%%%%%%%%%%%%%%%%%%%%%%%%%%%%%%%%%%%%%%%%%%%
%%%%%%%%%%%%%%%%%%%%%%%%%%%%%%%%%%%%%%%%%%%%%%%%%%%%%%%%%%%%%%%%%%%%%%%%%%%%%%%%


\subsection{Kilka słów o składni.}

Rozważmy język z rozdziału drugiego rozszerzony o następujące konstrukcje: \\

\begin{tabular}{| l c r |}
  \hline
  t ::= &  &  \\
   & $\textrm{let} \ x = t \ \textrm{in} \ t$ & \textit{zmienna lokalna} \\
   & $\textrm{tlet} \ X = T \ \textrm{in} \ t$ & \textit{typ lokalny} \\
   & & \\
  T ::= &  &  \\
   & $\bar{X}$ & \textit{zmienna typowa (kwantyfikowana schematem)} \\
   & & \\
  K ::= & & \\
   & $\widehat{X}$ & \textit{zmienna rodzajowa (kwantyfikowana schematem)} \\
   & & \\
  \hline
\end{tabular} \\

W dodatku zakładamy, że wszystkie konstruktorowe zmienne związane są unikatowe (tzn. zmienna $X$ może
występować za conajwyżej jednym kwantyfikatorem).

Ważnym założeniem jest to, że $\bar{X}::*$, co nam ułatwi sprawę przy unifikacji, a~zarazem zbytnio nie ograniczy języka.
  
Jeżeli chcemy dodać ML-polimorfizm, będą nam portrzebne schematy typów i schematy rodzajów:\\

\begin{tabular}{| l c r |}
  \hline
  $\left<T\right>$ ::= &  & \textit{schemat typu} \\
   & $T$ & \textit{typ schematu} \\
   & $\Omega \bar{X}.\left<T\right>$ & \textit{kwantyfikacja zmiennej typowej} \\
   & $\Omega \widehat{X}.\left<T\right>$ & \textit{kwantyfikacja zmiennej rodzajowej} \\
   & & \\
  $\left<K\right>$ ::= &  & \textit{schemat rodzaju} \\
   & $K$ & \textit{rodzaj schematu}\\
   & $\Omega \widehat{X}.\left<K\right>$ & \textit{kwantyfikacja zmiennej rodzajowej} \\
   & & \\
  \hline
\end{tabular} \\

Będziemy używać następującego cukru syntaktycznego:
\[
\Omega \bar{X}_1 \dots \bar{X}_n \widehat{X}_1 \dots \widehat{X}_m.T \equiv 
\Omega \bar{X}_1. \dots \Omega \bar{X}_n. \Omega \widehat{X}_1. \dots \Omega \widehat{X}_m.T
\]
I analogicznie dla rodzajów.

Teraz sam kontekst typowania ma postać:

\begin{tabular}{| l c r |}
  \hline
  $\Gamma$ ::= &  & \\
   & $\emptyset$ & \\
   & $\Gamma, x:\left<T\right>$ & \\
   & $\Gamma, X::\left<K\right>$ & \\
   & & \\
  \hline
\end{tabular} \\

Jeżeli chcemy spojrzeć na ten język, jak na język programowania z~rekonstrukcją typów, można
dopuścić nieanotowane kwantyfikatory, które parser będzie zamieniać na anotowane świeżą zmienną schematową 
(typową albo rodzajową w~zależności od kwantyfikatora). Z~drugiej strony nie ma potrzeby pozwalać programiście
jawnie korzystać ze zmiennych schematowych.

\subsection{Zbiory zmiennych i~podstawienia}

W~dalszej części będziemy używać różnych zbiorów zmiennych występujących w~konstrukcjach w~naszym języku, więc
to jest dobre miejsce by te zbiory i~operacje zdefiniować.

\subsubsection{Wszystkie zmienne schematowe}

\begin{definicja}
\emph{Zbiorem schematowych zmiennych typowych typu/schematu $T$} nazwiemy zbiór $\bar{Var}(T)$ zdefiniowany następująco
\begin{align*}
\bar{Var}(X) & = \emptyset \\
\bar{Var}(T_1 \to T_2) & = \bar{Var}(T_1) \cup \bar{Var}(T_2) \\
\bar{Var}(\forall X :: K.T) & = \bar{Var}(T) \\
\bar{Var}(\lambda X :: K.T) & = \bar{Var}(T) \\
\bar{Var}(T_1 \; T_2) & = \bar{Var}(T_1) \cup \bar{Var}(T_2) \\
\bar{Var}(\bar{X}) & = \{ \bar{X} \}
\end{align*}
\begin{align*}
\bar{Var}(\Omega \bar{X}.\left<T\right>) & = \{ \bar{X} \} \cup \bar{Var}(\left<T\right>) \\
\bar{Var}(\Omega \widehat{X}.\left<T\right>) & = \bar{Var}(\left<T\right>)
\end{align*}
\end{definicja}

\begin{definicja}
Anologicznie definijuemy \emph{zbiór zmiennych rodzajowych rodzaju/typu/schematu $T$}.
\begin{align*}
\widehat{Var}(*) & = \emptyset \\
\widehat{Var}(K_1 \Rightarrow K_2) & = \widehat{Var}(K_1) \cup \widehat{Var}(K_2) \\
\widehat{Var}(\widehat{X}) & = \{ \widehat{X} \}
\end{align*}
\begin{align*}
\widehat{Var}(X) & = \emptyset \\
\widehat{Var}(T_1 \to T_2) & = \widehat{Var}(T_1) \cup \widehat{Var}(T_2) \\
\widehat{Var}(\forall X :: K.T) & = \widehat{Var}(K) \cup \widehat{Var}(T) \\
\widehat{Var}(\lambda X :: K.T) & = \widehat{Var}(K) \cup \widehat{Var}(T) \\
\widehat{Var}(T_1 \; T_2) & = \widehat{Var}(T_1) \cup \widehat{Var}(T_2) \\
\widehat{Var}(\widehat{X}) & = \emptyset
\end{align*}
\begin{align*}
\widehat{Var}(\Omega \bar{X}.\left<T\right>) & = \widehat{Var}(\left<T\right>) \\
\widehat{Var}(\Omega \widehat{X}.\left<T\right>) & = \{ \widehat{X} \} \cup \widehat{Var}(\left<T\right>)
\end{align*}
\end{definicja}

\subsubsection{Zmienne typowe wolne i~związane}

\begin{definicja}
\emph{Zbiór zmiennych typowych wolnych termu/typu $T$} nazwiemy zbiór $FTV(T)$ zdefiniowany następująco
\begin{align*}
FTV(X) & = X \\
FTV(T_1 \to T_2) & = FTV(T_1) \cup FTV(T_2) \\
FTV(\forall X :: K . T) & = FTV(T) \setminus \{ X \} \\
FTV(\lambda X :: K . T) & = FTV(T) \setminus \{ X \} \\
FTV(T_1 \; T_2) & = FTV(T_1) \cup FTV(T_2) \\
FTV(\bar{X}) & = \emptyset \\
\end{align*}
\begin{align*}
FTV(x) & = \emptyset \\
FTV(\lambda x : T.t) & = FTV(T) \cup FTV(t) \\
FTV(t_1 \; t_2) & = FTV(t_1) \cup FTV(t_2) \\
FTV(\lambda X :: K.t) & = FTV(t) \setminus \{ X \} \\
FTV(t[T]) & = FTV(t) \cup FTV(T) \\
FTV(\textrm{let} \ x = t_1 \ \textrm{in} \ t_2) & = FTV(t_1) \cup FTV(t_2) \\
FTV(\textrm{tlet} \ X = T \ \textrm{in} \ t) & = FTV(T) \cup (FTV(t) \setminus \{ X \}) \\
\end{align*}
\end{definicja}

\begin{definicja}
\emph{Zbiór zmiennych typowych związanych termu/typu $T$} nazwiemy zbiór $BTV(T)$ zdefiniowany następująco
\begin{align*}
BTV(X) & = \emptyset \\
BTV(T_1 \to T_2) & = BTV(T_1) \cup BTV(T_2) \\
BTV(\forall X :: K . T) & = BTV(T) \cup \{ X \} \\
BTV(\lambda X :: K . T) & = BTV(T) \cup \{ X \} \\
BTV(T_1 \; T_2) & = BTV(T_1) \cup BTV(T_2) \\
BTV(\bar{X}) & = \emptyset \\
\end{align*}
\begin{align*}
BTV(x) & = \emptyset \\
BTV(\lambda x : T.t) & = BTV(T) \cup BTV(t) \\
BTV(t_1 \; t_2) & = BTV(t_1) \cup BTV(t_2) \\
BTV(\lambda X :: K.t) & = BTV(t) \cup \{ X \} \\
BTV(t[T]) & = BTV(t) \cup BTV(T) \\
BTV(\textrm{let} \ x = t_1 \ \textrm{in} \ t_2) & = BTV(t_1) \cup BTV(t_2) \\
BTV(\textrm{tlet} \ X = T \ \textrm{in} \ t) & = BTV(T) \cup BTV(t) \cup \{ X \} \\
\end{align*}
\end{definicja}

\subsubsection{Podstawienia}

Założenie o~unikalności zmiennych związanych wiąże się z~niewielką modyfkacją podstawienia.
\[
\{X := T\} Y = 
	\begin{cases}
		\alpha T & X=Y, \alpha \ \textrm{jest przemianowaniem zmiennych} \\
		& \textrm{związanych w~$T$ na świeże} \\
		Y & X \neq Y
	\end{cases} \\
\]
Reszta definicji pozostaje bez zmian.

Przy rekonstrukcji typów będziemy dodatkowo używać podstawienia, które dodatkowo wstawia świeże zmienne rodzajowe.
Zdefiniujemy je podobnie:
\[
\{X(\{\widehat{X}_1, \dots, \widehat{X}_n\}) := T\} Y = 
	\begin{cases}
		\alpha \sigma T & X=Y, \alpha \ \textrm{jest przemianowaniem zmiennych} \\
		& \textrm{związanych w~$T$ na świeże}, \\
		& \sigma = [\widehat{X}_1 := fresh, \dots, \widehat{X}_n := fresh] \\
		Y & X \neq Y
	\end{cases} \\
\]
\[
\ldots
\]
$\sigma$ jest podstawieniem schematowym, definicja jego działania pojawi się później.

Purystów matematycznych może przerażać pojawiająca się w~definicji ,,świeża zmienna'', oznaczana
jako $fresh$. Jest to zmienna która dotychczas nigdzie się nie pojawiła, i~nie będzie wprowadzona więcej niż raz.
Można ją formalnie zdefiniować pamiętając cały czas zbiór użytych zmiennych i~przekazując go jako dodatkowy argument do prawie
wszystkich operacji, ale spowodowało by to znaczne pogorszenie czytelności tego dokumentu, dlatego tak nie robimy.
Zainteresowany Czytelnik może samodzielnie poprawić przytoczone tu definicje i~twierdzenia formalnie definiując świeżą zmienną.

Dla przytoczonych tu definicji później zajdzie potrzeba rozszerzenia ich na konteksty i~podstawienia.
Rozszerzenia te są na tyle oczywiste i~naturalne, że nie ma potrzeby ich przytaczania.

\subsection{$\beta$-unifikacja}

Podczas rekonstukcji typów pojawiają się równania więzów które należy rozwiązać. Samo rozwiązanie
sprowadza się do unifikacji pewnych termów, ale tutaj, ze względu na możliwość występowania funkcji
typowych, termy równoważne nie muszą być równe, więc sama unifikacja powinna sprowadzać termy do
$\beta$-równych sobie.

Taka unifikacja niesie ze sobą wiele problemów. Po pierwsze, będziemy chcieli używać podstawienia (za zmienne schematowe),
będącego unifikatorem, na innych termach. To zaś grozi uzewnętrznieniem zmiennych związanych. Drugi
problem to taki, że samo podstawienie nie musi zrównywać termów, które da się zrównać, ale różnią się
nazwami zmiennych związanych.

Z pierwszym problemem poradzimy sobie, traktując podstawienia jako funkcję częściową, tzn. samo podstawienie
oprócz przyporządkowań postaci $[\bar{X} := T]$, może również zawierać podstawienia postaci $[\bar{X} := fail]$.
Dodając operację anulowania podstawienia otrzymujemy:
\[
[\bar{X} := T] \setminus X = 
\begin{cases}
[\bar{X} := fail] & X \in    FTV(T) \\
[\bar{X} := T]    & X \notin FTV(T)
\end{cases}
\]

Z drugim problemem poradzimy sobie traktując unifikator jako parę zawierającą podstawienie, które dobrze działa
na zewnątrz, oraz term będący wynikiem unifikacji.

\begin{definicja}
\emph{Podstawieniem schematowym} nazwiemy skończony zbiór par postaci $(\bar{X},T)$, $(\bar{X},fail)$ oraz $(\widehat{X},K)$, takich że
$\bar{X} \notin \bar{Var}(T)$ oraz $\widehat{X} \notin \widehat{Var}(K)$ (po podstawieniu powinny zniknąć wszystkie zmienne, za 
które coś podstawiliśmy). 
\emph{Podstawieniem schematowym pustym} nazwiemy podstawienie schematowe będące zbiorem pustym i~będziemy oznaczać przez $[]$.
Podstawienia schematowe będziemy reprezentować jako listy postaci
\[
[\bar{X}_1 := T_1, \dots, \bar{X}_k := T_k, \bar{X}_{k+1} := fail, \dots , \bar{X}_n := fail, \widehat{X}_1 := K_1, \dots, \widehat{X}_m := K_m].
\]
\emph{Dziedziną} podstawienia schematowego $\sigma$ nazwiemy zbiór takich zmiennych typowych $\bar{X}$ i~rodzajowych $\widehat{X}$, że
$(\bar{X}, T) \in \sigma$ lub $(\bar{X}, fail) \in \sigma$ oraz $(\widehat{X}, K) \in \sigma$. Dziedzinę podstawienia schematowego $\sigma$
będziemy oznaczać przez $Dom(\sigma)$.
\end{definicja}

\begin{definicja}
Podstawienie schematowe może działać na rodzajach, typach, termach i~kontekstach. Jeżeli w~podstawieniu występuje
para $\widehat{X} := K$, to za wszystkie wystąpienia zmiennej~$\widehat{X}$ zostanie podstawione $K$. Analogicznie
się dzieje w~przypadku wystąpienia pary $\bar{X} := T$. Jeżeli w~podstawieniu schematowym wystąpi para $\bar{X} := fail$,
a~w~typie/termie/kontekście do którego podstawienie aplikujemy występuje zmienna $\bar{X}$, to podstawienie jest niemożliwe.

Niech $\sigma$ będzie podstawieniem schematowym. Działanie podstawienia formalnie definiujemy jako:
\begin{align*}
\sigma * & = * \\
\sigma (K_1 \Rightarrow K_2) & = \sigma K_1 \Rightarrow \sigma K_2 \\
\sigma \widehat{X} & = 
	\begin{cases}
		K & (\widehat{X},K) \in \sigma \\
		\widehat{X} & \textrm{wpp}
	\end{cases} \\
\end{align*}
\begin{align*}
\sigma X & = X \\
\sigma(T_1 \to T_2) & = \sigma T_1 \to \sigma T_2 \\
\sigma(\forall X :: K.T) & = \forall X :: \sigma K. \sigma T \\
\sigma(\lambda X :: K.T) & = \lambda X :: \sigma K. \sigma T \\
\sigma(T_1 \; T_2) & = (\sigma T_1) \; (\sigma T_2) \\
\sigma \bar{X} & = 
	\begin{cases}
		\alpha T & (\bar{X},T) \in \sigma, \alpha \ \textrm{jest przemianowaniem} \\
		& \textrm{wprowadzającym świeże zmienne związane} \\
		fail & (\bar{X},fail) \in \sigma \\
		\bar{X} & \textrm{wpp}
	\end{cases} \\
\end{align*}
\begin{align*}
\sigma x & = x \\
\sigma (\lambda x : T.t) & = \lambda x : \sigma T. \sigma t \\
\sigma (t_1 \; t_2) & = (\sigma t_1) \; (\sigma t_2) \\
\sigma (\lambda X :: K. t) & = \lambda X :: \sigma X . \sigma t \\
\sigma (t[T]) & = (\sigma t)[\sigma T] \\
\sigma (\textrm{let} \ x = t_1 \ \textrm{in} \ t_2) & = \textrm{let} \ x = \sigma t_1 \ \textrm{in} \ \sigma t_2 \\
\sigma (\textrm{tlet} \ x = T \ \textrm{in} \ t) & = \textrm{tlet} \ x = \sigma T \ \textrm{in} \ \sigma t \\
\end{align*}
\begin{align*}
\sigma (\Omega\bar{X}.\left<T\right>) & =
	\begin{cases}
		\Omega\bar{X}.\sigma\left<T\right> & \bar{X} \notin Dom(\sigma) \\
		\Omega\bar{Y}.\sigma[\bar{X}:=\bar{Y}]\left<T\right> & \bar{X} \in Dom(\sigma) \land \bar{Y} \notin Dom(\sigma) \land \bar{Y} \notin \bar{Var}(\left<T\right>)
	\end{cases} \\
\sigma (\Omega\widehat{X}.\left<T\right>) & =
	\begin{cases}
		\Omega\widehat{X}.\sigma\left<T\right> & \widehat{X} \notin Dom(\sigma) \\
		\Omega\widehat{Y}.\sigma[\widehat{X}:=\widehat{Y}]\left<T\right> & \widehat{X} \in Dom(\sigma) \land \widehat{Y} \notin Dom(\sigma) \land \widehat{Y} \notin \widehat{Var}(\left<T\right>)
	\end{cases}
\end{align*}
\begin{align*}
\sigma \emptyset & = \emptyset \\
\sigma (\Gamma, x : \left<T\right>) & = \sigma \Gamma, x : \sigma \left<T\right> \\
\sigma (\Gamma, X :: \left<K\right>) & = \sigma \Gamma, X :: \sigma \left<K\right> \\
\end{align*}
\end{definicja}

\begin{definicja}
Powiemy, że podstawienie schematowe $\rho$ jest \emph{złożeniem} podstawień schematowych $\sigma$ i~$\theta$, 
które będziemy oznaczać $\sigma\circ\theta$ jeżeli
dla każdego rodzaju $K$, typu $T$, termu $t$ i~kontekstu $\Gamma$ zachodzi:
\begin{align*}
\rho K & = \sigma(\theta K) \\
\rho T & = \sigma(\theta T) \\
\rho t & = \sigma(\theta t) \\
\rho \Gamma & = \sigma(\theta \Gamma)
\end{align*}
\end{definicja}

\begin{lemat}
Niech $\sigma$ i~$\theta$ będą podstawieniami schematowymi. Wówczas zachodzi
\begin{align*}
\sigma \circ \theta = & \{ (\widehat{X}, \sigma K) | (\widehat{X}, K) \in \theta \} \\
\cup & \{ (\widehat{X}, K) \in \sigma | \widehat{X} \notin Dom(\theta) \} \\
\cup & \{ (\bar{X}, \sigma T) | (\bar{X}, T) \in \theta \land \sigma T \neq fail \} \\
\cup & \{ (\bar{X}, fail) | \exists (\bar{X}, T) \in \theta . \sigma T = fail \} \\
\cup & \{ (\bar{X}, T) \in \sigma | \bar{X} \notin Dom(\theta) \} \\
\cup & \{ (\bar{X}, fail) \in \sigma | \bar{X} \notin Dom(\theta) \}
\end{align*}
Pozwala nam to algorytmicznie wyliczać złożenia podstawień schematowych.
\end{lemat}


\begin{definicja}
\emph{Anulowaniem zbioru $A$ zmiennych konstruktorowych z~podstawienia schematowego $\sigma$} nazwiemy operację
zdefiniowaną następująco:
\begin{align*}
\sigma \setminus A = & \{ (\widehat{X}, K) \in \sigma \} \cup \{ (\bar{X}, fail) \in \sigma \} \\
\cup & \{ (\bar{X}, T) \in \sigma | FTV(T) \cap A = \emptyset \} \\
\cup & \{ (\bar{X}, fail) | \exists (\bar{X}, T) \in \sigma . FTV(T) \cap A \neq \emptyset \}
\end{align*}
\emph{Anulowanie zmiennej $X$ z~podstawienia schematowego $\sigma$} oznaczamy przez $\sigma \setminus X$ i~definiujemy jako $\sigma \setminus \{X\}$.
\end{definicja}

\begin{definicja}
\emph{$\beta$-unifikatorem} dla konstruktorów typów $T_1$ i $T_2$ nazwiemy taką parę $(\sigma, S)$, że istnieje
podstawienie schematowe $\theta$ i~przemianowania zmiennych związanych $\alpha_1$ i~$\alpha_2$ takie, że:
\[
\theta \alpha_1 T_1 =_\beta S =_\beta \theta \alpha_2 T_2
\]
oraz
\[
\theta \setminus BTV(S) = \sigma
\]
\end{definicja}

\begin{definicja}
Powiemy, że $\beta$-unifikator $(\sigma, S)$ jest \emph{ogólniejszy} od $\beta$-unifikatora
$(\theta, T)$, jeżeli istnieje takie podstawienie schematowe $\rho$ i~przemianowanie $\alpha$, że
\[
\rho\alpha S =_\beta T \quad \textrm{oraz} \quad \rho\alpha\sigma = \theta
\]
\end{definicja}

\begin{definicja}
\emph{Najogólniejszym $\beta$-unifikatorem} dla konstruktorów typów $T_1$ i $T_2$ nazwiemy taki $\beta$-unifikator,
który jest ogólniejszy od wszystkich innych $\beta$-unifikatorów tychże konstruktorów typów.
\end{definicja}

\begin{lemat}
Niech $A$ i $B$ będą podzbiorami zbioru zmiennych konstruktorowych, niech $\theta$ będzie podstawieniem schematowym.
Wówczas zachodzi
\[
(\theta \setminus A) \setminus B = \theta \setminus (A \cup B).
\]
\label{lsub0}
\end{lemat}

\begin{lemat}
Niech $A$ będzie podzbiorem zbioru zmiennych konstruktorowych, niech $\theta$ i~$\rho$ będą podstawieniami schematowymi, oraz
niech zachodzi $FTV(\rho) \cap A = \emptyset$. Wówczas zachodzi 
\[
(\rho \circ \theta) \setminus A = \rho \circ (\theta \setminus A)
\].
\label{lsub1}
\end{lemat}

\begin{lemat}
Niech $A$ będzie podzbiorem zbioru zmiennych konstruktorowych, niech $\theta$ będzie dowolnym podstawieniem schematowym,
niech $\rho$ będzie podstawieniem schematowym takim, że $FTV(\rho) \cap A = \emptyset$ (w~szczególności zawierającym tylko zmienne rodzajowe). 
Wówczas zachodzi
\[
(\theta \circ \rho) \setminus A = (\theta \setminus A) \circ \rho
\]
\label{lsub2}
\end{lemat}

\begin{lemat}
Niech $(\sigma, S)$ będzie $\beta$-unifikatorem dla konstruktorów typów $T_1$ i~$T_2$, oraz niech $\rho$ będzie 
podstawieniem schematowym takim, że \mbox{$FTV(\rho) \cap BTV(S) = \emptyset$}.
Wówczas $(\rho \circ \sigma, \rho S)$ również jest $\beta$-unifikatorem dla konstruktorów typów $T_1$ i~$T_2$.
\label{lunif1}
\end{lemat}
\begin{proof}
Wiemy, że istnieją takie $\theta$, $\alpha_1$ i~$\alpha_2$, że
\[
\theta \alpha_1 T_1 =_\beta S =_\beta \theta \alpha_2 T_2 \quad \textrm{oraz} \quad \theta \setminus BTV(S) = \sigma
\]
A wtedy zachodzi
\[
(\rho \circ \theta) \alpha_1 T_1 = \rho(\theta \alpha_1 T_1) =_\beta \rho S =_\beta \rho(\theta \alpha_2 T_2) = (\rho \circ \theta_2) \alpha_2 T_2
\]
oraz z lematu \ref{lsub1}
\[
(\rho \circ \theta) \setminus BTV(S) = \rho \circ (\theta \setminus BTV(S)) = \rho \circ \sigma
\]
\end{proof}

\begin{lemat}
Niech $T_1$ i~$T_2$ będą konstruktorami typów, $\rho$ niech będzie podstawieniem schematowym takim, że 
$FTV(\rho) \cap BTV(T_1) = \emptyset$ oraz $FTV(\rho) \cap BTV(T_2) = \emptyset$, a $(\sigma, S)$ niech będzie 
\mbox{$\beta$-unifikatorem} konstruktorów $(\rho T_1, \rho T_2)$.
Wówczas $(\sigma \circ \rho, S)$ jest $\beta$-unifikatorem konstruktorów $(T_1, T_2)$.
\label{lunif2}
\end{lemat}
\begin{proof}
Niech $i \in \left\{1,2\right\}$.\\
Z tego, że $(\sigma, S)$ jest $\beta$-unifikatorem konstruktorów $(\rho T_1, \rho T_2)$ wiemy, że istnieje takie
$\theta$ oraz $\alpha_i$, że
\[
\theta \alpha_i \rho T_i =_\beta S \quad \textrm{oraz} \quad \theta \setminus BTV(S) = \sigma
\]
A~wtedy z~tego, że $FTV(\rho) \cap BTV(T_i) = \emptyset$
\[
(\theta \circ \rho) \alpha_i T_i = \theta \alpha_i \rho T_i =_\beta S
\]
oraz z~lematu \ref{lsub2} mamy
\[
(\theta \circ \rho) \setminus BTV(S) = (\theta \circ BTV(S)) \circ \rho = \sigma \circ \rho
\]
Co kończy dowód.
\end{proof}

\subsubsection{Algorytm $\beta$-unifikacji}

Przy unifikacji wszystkie nieanotowane kwantyfikatory traktujemy jak anotowane unikatową zmienną.
Funkcja $unify$ to klasyczna unifikacja (na rodzajach).\\
\begin{tabular}{l}
$unify_\beta(T_1, T_2) \ \textrm{when} \ T_1 =_\beta T_2 \ = $ \\
$\qquad ([], T_1)$ \\
\end{tabular} \\
\begin{tabular}{l}
$unify_\beta(\bar{X}, T) = $ \\
$\qquad \textrm{if} \ \bar{X} \in \bar{Var}(T) \ \textrm{then} \ fail $ \\
$\qquad \textrm{else} \ ([\bar{X} := T], T) $ \\
\end{tabular} \\
\begin{tabular}{l}
$unify_\beta(T, \bar{X}) = $ \\
$\qquad \textrm{if} \ \bar{X} \in \bar{Var}(T) \ \textrm{then} \ fail $ \\
$\qquad \textrm{else} \ ([\bar{X} := T], T) $ \\
\end{tabular} \\
\begin{tabular}{l}
$unify_\beta(\forall X_1::K_1.T_1, \forall X_2::K_2.T_2) = $ \\
$\qquad \textrm{let} \ \sigma_K = unify(K_1, K_2) \ \textrm{in} $ \\
$\qquad \textrm{let} \ X_3 = fresh \ \textrm{in} $ \\
$\qquad \textrm{let} \ (\sigma_T, T) = unify_\beta(\sigma_K \sigma_K \{X_1 := X_3\}T_1, \sigma_K \{X_2 := X_3\}T_2) \ \textrm{in} $ \\
$\qquad\qquad ((\sigma_T \setminus X_3) \circ \sigma_K, \forall X_3::\sigma_T \sigma_K K_1.T) $ \\
\end{tabular} \\
\begin{tabular}{l}
$unify_\beta(\lambda X_1::K_1.T_1, \lambda X_2::K_2.T_2) = $ \\
$\qquad \textrm{let} \ \sigma_K = unify(K_1, K_2) \ \textrm{in} $ \\
$\qquad \textrm{let} \ X_3 = fresh \ \textrm{in} $ \\
$\qquad \textrm{let} \ (\sigma_T, T) = unify_\beta(\sigma_K \{X_1 := X_3\}T_1, \sigma_K \{X_2 := X_3\}T_2) \ \textrm{in} $ \\
$\qquad\qquad ((\sigma_T \setminus X_3) \circ \sigma_K, \lambda X_3::\sigma_T \sigma_K K_1.T) $ \\
\end{tabular} \\
\begin{tabular}{l}
$unify_\beta(T_1 \rightarrow S_1, T_2 \rightarrow S_2) = $ \\
$\qquad \textrm{let} \ (\theta, T) = unify_\beta(T_1, T_2) \ \textrm{in} $ \\
$\qquad \textrm{let} \ (\sigma, S) = unify_\beta(\theta S_1, \theta S_2) \ \textrm{in} $ \\
$\qquad\qquad (\sigma \circ \theta, \sigma T \rightarrow S) $ \\
\end{tabular} \\
\begin{tabular}{l}
$unify_\beta(T_1, T_2) \ \textrm{when} \ T_1 \longrightarrow_\beta S_1 \ = $ \\
$\qquad unify_\beta(S_1, T_2)$ \\
\end{tabular} \\
\begin{tabular}{l}
$unify_\beta(T_1, T_2) \ \textrm{when} \ T_2 \longrightarrow_\beta S_2 \ = $ \\
$\qquad unify_\beta(T_1, S_2)$ \\
\end{tabular} \\
\begin{tabular}{l}
$unify_\beta(V_1 \; T_1, V_2 \; T_2) = $ \\
$\qquad \textrm{let} \ (\theta, V) = unify_\beta(V_1, V_2) \ \textrm{in} $ \\
$\qquad \textrm{let} \ (\sigma, T) = unify_\beta(\theta T_1, \theta T_2) \ \textrm{in} $ \\
$\qquad\qquad (\sigma \circ \theta, \sigma V \; T) $ \\
\end{tabular} \\

\subsubsection{Własności}

\begin{fakt}
Algorytm $unify_\beta$ dla konstruktorów typów, dla których daje się określić rodzaj, zatrzymuje się.
\end{fakt}
\begin{proof}
Trywialne. Wynika z~tego, że algorytm jest sterowany składnią, oraz z~silnej normalizowalności kostruktorów typów.
\end{proof}
\begin{twierdzenie}
Niech $T_1$ oraz $T_2$ będą konstruktorami typów, które daje się orodzajować. Wówczas jeśli
$unify_\beta(T_1, T_2) = (\sigma, S)$, to $(\sigma, S)$ jest $\beta$-unifikatorem konstruktorów typów
$T_1$ i~$T_2$.
\end{twierdzenie}
\begin{proof}
Indukcja po głebokości rekursji.

Możliwe są następujące przypadki:
\begin{itemize}
\item $T_1 =_\beta T_2$ lub $T_1 \equiv \bar{X}$ lub $T_2 \equiv \bar{X}$ \\
dla tych przypadków teza zachodzi trywialnie.
\item $T_1 \equiv \forall X_1::K_1.T'_1$ oraz $T_2 \equiv \forall X_2::K_2.T'_2$ \\
Wtedy z założenia indukcyjnego $(\sigma_T, T)$ jest $\beta$-unifikatorem konstruktorów typów
$\sigma_K \{X_1:=X_3\}T'_1$ oraz $\sigma_K\{X_2:=X_3\}T'_2$, gdzie $\sigma_K$ jest unifikatorem rodzajów $K_1$ i~$K_2$.
Istnieje więc takie podstawienie schematowe $\theta$ oraz przemianowania $\alpha_1$ i~$\alpha_2$, że zachodzą równości
\[
\theta \alpha_1 \sigma_K \{X_1:=X_3\}T'_1 =_\beta T =_\beta \theta \alpha_2 \sigma_K \{X_2:=X_3\} T'_2
\]
\[
\theta \setminus BTV(T) = \sigma_T
\]
A wtedy
\[
(\theta \circ \sigma_K)\alpha_1\{X_1:=X_3\} \forall X_1::K_1.T'_1 = \forall X_3::\theta\sigma_K K_1. \theta\sigma_K\alpha_1 \{X_1 := X_3\} T'_1 =
\]
\[
= \forall X_3::(\theta \setminus BTV(T))\sigma_K K_1.\theta \alpha_1 \sigma_K \{X_1 := X_3\}T'_1 =_\beta \forall X_3::\sigma_T\sigma_K K_1.T
\]
oraz
\[
(\theta \circ \sigma_K) \setminus BTV(\forall X_3::\sigma_T\sigma_K K_1.T) = (\theta \setminus (\{X_3\} \cup BTV(T))) \circ \sigma_K =
\]
\[
= (\sigma_T \setminus X_3) \circ \sigma_K
\]
Analogicznie pokażemy pozostałe dwie równości dla konstruktora $T_2$.\\ Zatem 
\mbox{$((\sigma_T \setminus X_3) \circ \sigma_K, \forall X_3::\sigma_T \sigma_K K_1.T)$} jest $\beta$-unifikatorem konstruktorów $T_1$ i~$T_2$.
\item $T_1 \equiv \lambda X_1::K_1.T'_1$ oraz $T_2 \equiv \lambda X_2::K_2.T'_2$ \\
Dla tego przypadku dowód przeprowadzamy analogicznie do przypadku poprzedniego.
\item $T_1 \equiv T'_1 \rightarrow S_1$ oraz $T_2 \equiv T'_2 \rightarrow S_2$ \\
Teza wynika wprost z~założenia indukcyjnego oraz lematów \ref{lunif1} i \ref{lunif2}.
\item $T_1 \longrightarrow_\beta S_1$ lub $T_2 \longrightarrow_\beta S_2$ \\
Wynika z~założenia indukcyjnego i~z~faktu, że $\beta$-redukcja zachowuje $\beta$-równość.
\item $T_1 \equiv V_1\;S_1$ oraz $T_2 \equiv V_2\;S_2$ \\
Dowód przeprowadzamy analogicznie do przypadku z typem funkcji.
\end{itemize}
\end{proof}

\begin{fakt}
Algorytm $unify_\beta$ działający na konstruktorach typów nie zawierających zmiennych schematowych
zwraca pusty unifikator jeśli konstruktory są $\beta$ równe, w~przeciwnym przypadku zawodzi (zawraca $fail$).
\end{fakt}


\begin{fakt}
Jeżeli w~algorytmie pierwszy przypadek sprawdzający $\beta$-równość konstruktorów zamienimy na analogiczy 
sprawdzający tożsamość zmiennych typowych (a~przy rozszerzeniach jeszcze typów bazowych), otrzymamy algorytm
równoważny.
\end{fakt}


\subsection{Rekonstrukcja typów}

\subsubsection{Algorytm W}

Dla rodzajów: \\
\begin{tabular}{l}
$kindof(\Gamma \vdash X :: ?) = $ \\
$\qquad \textrm{let} \ \Omega \widehat{X}_1 \dots \widehat{X}_n.K = \Gamma(X) \ \textrm{in} $ \\
$\qquad \textrm{let} \ \widehat{Y}_1, \dots, \widehat{Y}_n = fresh \ \textrm{in} $ \\
$\qquad\qquad ([], [\widehat{X}_1 := \widehat{Y}_1, \dots ,\widehat{X}_n := \widehat{Y}_n]K) $ \\
\end{tabular} \\
\begin{tabular}{l}
$kindof(\Gamma \vdash T_1 \rightarrow T_2 :: ?) = $ \\
$\qquad \textrm{let} \ (\sigma_1, K_1) = kindof(\Gamma \vdash T_1 :: ?) \ \textrm{in} $ \\
$\qquad \textrm{let} \ (\sigma_2, K_2) = kindof(\sigma_1 \Gamma \vdash \sigma_1 T_1 :: ?) \ \textrm{in} $ \\
$\qquad \textrm{let} \ \rho_1 = unify(\sigma_2 K_1, *) \ \textrm{in} $ \\
$\qquad \textrm{let} \ \rho_2 = unify(\rho_1 K_2, *) \ \textrm{in} $ \\
$\qquad\qquad (\rho_2 \circ \rho_1 \circ \sigma_2 \circ \sigma_1, *) $ \\
\end{tabular} \\
\begin{tabular}{l}
$kindof(\Gamma \vdash \forall X::K.T :: ?) = $ \\
$\qquad \textrm{let} \ (\sigma, K_T) = kindof(\Gamma, X::K \vdash T :: ?) \ \textrm{in} $ \\
$\qquad \textrm{let} \ \rho = unify(K_T, *) \ \textrm{in} $ \\
$\qquad\qquad (\rho \circ \sigma, *) $ \\
\end{tabular} \\
\begin{tabular}{l}
$kindof(\Gamma \vdash \lambda X::K.T :: ?) = $ \\
$\qquad \textrm{let} \ (\sigma, K_T) = kindof(\Gamma, X::K \vdash T :: ?) \ \textrm{in} $ \\
$\qquad\qquad (\sigma, \sigma K \Rightarrow K_T) $ \\
\end{tabular} \\
\begin{tabular}{l}
$kindof(\Gamma \vdash T_1 \; T_2 :: ?) = $ \\
$\qquad \textrm{let} \ (\sigma_1, K_1) = kindof(\Gamma \vdash T_1 :: ?) \ \textrm{in} $ \\
$\qquad \textrm{let} \ (\sigma_2, K_2) = kindof(\sigma_1 \Gamma \vdash \sigma_1 T_2 :: ?) \ \textrm{in} $ \\
$\qquad \textrm{let} \ \widehat{X} = fresh \ \textrm{in} $ \\
$\qquad \textrm{let} \ \sigma_3 = unify(\sigma_2 K_1, K_2 \Rightarrow \widehat{X}) \ \textrm{in} $ \\
$\qquad\qquad (\sigma_3 \circ \sigma_2 \circ \sigma_1, \sigma_3 \widehat{X}) $ \\
\end{tabular} \\
\begin{tabular}{l}
$kindof(\Gamma \vdash \bar{X} :: ?) = $ \\
$\qquad\qquad ([], *) $ \\
\end{tabular} \\

Dla typów:\\
\begin{tabular}{l}
$typeof(\Gamma \vdash x : ?) = $ \\
$\qquad \textrm{let} \ \Omega \bar{X}_1 \dots \bar{X}_n \widehat{X}_1 \dots \widehat{X}_m.T = \Gamma(x) \ \textrm{in} $ \\
$\qquad \textrm{let} \ \bar{Y}_1, \dots, \bar{Y}_n, \widehat{Y}_1, \dots, \widehat{Y}_m = fresh \ \textrm{in} $ \\
$\qquad\qquad ([], [\bar{X}_1 := \bar{Y}_1, \dots ,\bar{X}_n := \bar{Y}_n,\widehat{X}_1 := \widehat{Y}_1, \dots ,\widehat{X}_m := \widehat{Y}_m]T) $ \\
\end{tabular} \\
\begin{tabular}{l}
$typeof(\Gamma \vdash \lambda x:T.t : ?) = $ \\
$\qquad \textrm{let} \ (\sigma_1, K) = kindof(\Gamma \vdash T :: ?) \ \textrm{in} $ \\
$\qquad \textrm{let} \ \sigma_2 = unify(K, *) \ \textrm{in} $ \\
$\qquad \textrm{let} \ (\sigma_3, T_t) = typeof(\sigma_2 \sigma_1(\Gamma, x:T) \vdash \sigma_2 \sigma_1 t : ?) \ \textrm{in} $ \\
$\qquad\qquad (\sigma_3 \circ \sigma_2 \circ \sigma_1, \sigma_3 \sigma_2 \sigma_1 T \rightarrow T_t) $ \\
\end{tabular} \\
\begin{tabular}{l}
$typeof(\Gamma \vdash t_1 \; t_2 : ?) = $ \\
$\qquad \textrm{let} \ (\sigma_1, T_1) = typeof(\Gamma \vdash t_1 : ?) \ \textrm{in} $ \\
$\qquad \textrm{let} \ (\sigma_2, T_2) = typeof(\sigma_1 \Gamma \vdash \sigma_1 t_2 : ?) \ \textrm{in} $ \\
$\qquad \textrm{let} \ \bar{X} = fresh \ \textrm{in} $ \\
$\qquad \textrm{let} \ (\sigma_3, T'_2 \to T_3) = unify_\beta(\sigma_2 T_1, T_2 \rightarrow \bar{X}) \ \textrm{in} $ \\
$\qquad\qquad (\sigma_3 \circ \sigma_2 \circ \sigma_1, T_3) $ \\
\end{tabular} \\
\begin{tabular}{l}
$typeof(\Gamma \vdash \lambda X::K.t : ?) = $ \\
$\qquad \textrm{let} \ (\sigma, T) = typeof(\Gamma, X::K \vdash t : ?) \ \textrm{in} $ \\
$\qquad\qquad (\sigma \setminus X, \forall X::\sigma K.T) $ \\
\end{tabular} \\
\begin{tabular}{l}
$typeof(\Gamma \vdash t[T] : ?) = $ \\
$\qquad \textrm{let} \ (\sigma_1, T_t) = typeof(\Gamma \vdash t : ?) \ \textrm{in} $ \\
$\qquad \textrm{let} \ (\sigma_2, K) = kindof(\sigma_1 \Gamma \vdash \sigma_1 T :: ?) \ \textrm{in} $ \\
$\qquad \textrm{let} \ \bar{X} = fresh \ \textrm{in} $ \\
$\qquad \textrm{let} \ Y = fresh \ \textrm{in} $ \\
$\qquad \textrm{let} \ (\sigma_3, \forall Z :: K' . T') = unify_\beta(\sigma_2 T_t, \forall Y::K.\bar{X}) \textrm{in} $ \\
$\qquad\qquad (\sigma_3 \circ \sigma_2 \circ \sigma_1, \{Y:=\sigma_3 \sigma_2 \sigma_1 T\} T') $ \\
\end{tabular} \\
\begin{tabular}{l}
$typeof(\Gamma \vdash \textrm{let} \ x=t_1 \ \textrm{in} \ t_2 : ?) = $ \\
$\qquad \textrm{let} \ (\sigma_1, T_1) = typeof(\Gamma \vdash t_1 : ?) \ \textrm{in} $ \\
$\qquad \textrm{let} \ \left\{ \bar{X}_1, \dots, \bar{X}_n \right\} = \bar{Var}(T_1) \setminus \bar{Var}(\Gamma) \ \textrm{in} $ \\
$\qquad \textrm{let} \ \left\{ \widehat{X}_1, \dots, \widehat{X}_m \right\} = \widehat{Var}(T_1) \setminus \widehat{Var}(\Gamma) \ \textrm{in} $ \\
$\qquad \textrm{let} \ (\sigma_2, T_2) = typeof(\sigma_1 \Gamma, x:\Omega \bar{X}_1 \dots \bar{X}_n \widehat{X}_1 \dots \widehat{X}_m. T_2 \vdash \sigma_1 t_2 : ?) \ \textrm{in} $ \\
$\qquad\qquad (\sigma_2 \circ \sigma_1, T_2) $ \\
\end{tabular} \\
\begin{tabular}{l}
$typeof(\Gamma \vdash \textrm{tlet} \ X=T \ \textrm{in} \ t : ?) = $ \\
$\qquad \textrm{let} \ (\sigma_1, K) = kindof(\Gamma \vdash T :: ?) \ \textrm{in} $ \\
$\qquad \textrm{let} \ (\sigma_2, T_t) = typeof(\sigma_1 \Gamma \vdash \{ X(\widehat{Var}(K) \setminus \widehat{Var}(\Gamma)) := \sigma_1 T \} \sigma_1 t : ?) \ \textrm{in} $ \\
$\qquad\qquad (\sigma_2 \circ \sigma_1, T_t) $ \\
\end{tabular} \\

\subsubsection{Własności}

\begin{twierdzenie}
Funkcja $typeof$ zatrzymuje się.
\end{twierdzenie}

\begin{twierdzenie}
Niech $t$ i~$\Gamma$ nie zawierają zmiennych schematowych. Wówczas \mbox{$typeof(\Gamma \vdash t : ?) = (\sigma, T)$} wtedy 
i~tylko wtedy, gdy $\Gamma \vdash t : T'$, gdzie $T' =_\beta T$ oraz $\sigma = []$ (tzn. gdy program zawiera wszystkie anotacje, algorytm $typeof$ wykonuje zwykłą 
kontrolę typów w~systemie $F_\omega$).
\label{twtypecheck}
\end{twierdzenie}
\begin{proof}
Szkic dowodu:
\begin{enumerate}
\item Indukcyjnie po strukturze typu pokazujemy analogiczne twierdzenie dla rodzajów i~funkcji $kindof$:\\
Niech $T$ i~$\Gamma$ nie zawierają zmiennych schematowych. Wówczas \mbox{$kindof(\Gamma \vdash t :: ?) = (\sigma, K)$} wtedy
i~tylko wtedy gdy $\Gamma \vdash T :: K$ oraz $\sigma = []$.
\item Indukcyjnie po strukturze wyrażenia $t$ pokażemy implikację tezy w~jedną stronę.
\item Pokazujemy indukcyjnie, że drzewa wyprowadzenia $\Gamma \vdash t : T$ można zawsze przebudować tak, by
z~reguły
\[
\frac{\Gamma \vdash t : S \quad S \equiv T \quad \Gamma \vdash T :: *}{\Gamma \vdash t : T}
\]
korzystać jedynie nad argumentem aplikacji i~w~korzeniu drzewa wyprowadzenia.
\item Indukcyjnie po tak przebudowanym drzewie pokażemy implikację tezy w~drugą stronę.
\end{enumerate}
\end{proof}

\begin{twierdzenie}
Niech $t$ i~$\Gamma$ będą takim termem i~kontekstem, że $typeof(\Gamma \vdash t : ?) = (\sigma, T)$. Wówczas
istnieje takie podstawienie schematowe $\theta$, że zachodzą następujące własności:
\begin{itemize}
\item $\theta \Gamma \vdash \theta t : T$
\item $\theta \setminus BTV(t) = \sigma$
\end{itemize}
\end{twierdzenie}
\begin{proof}
Szkic dowodu:
\begin{enumerate}
\item Podobnie jak w~dowodzie twierdzenia \ref{twtypecheck} najpierw indukcyjnie dowodzimy analogicznego twierdzenia
dla funkcji $kindof$ i~relacji rodzajowania.
\item Indukcyjnie po strukturze termu $t$ pokazujemy tezę. Podstawienie schematowe $\theta$ konstruujemy tak jak podstawienie schematowe
$\sigma$ tyle, że z~pominięciem anulowania zmiennych związanych. W~miejscach w~których korzystamy z~$\beta$-unifikacji
korzystamy z~definicji \mbox{$\beta$-unifikatora} by skonstruować podstawienie schematowe $\theta$.
\end{enumerate}
\end{proof}


%%%%%%%%%%%%%%%%%%%%%%%%%%%%%%%%%%%%%%%%%%%%%%%%%%%%%%%%%%%%%%%%%%%%%%%%%%%%%%%%
%%%%%%%%%%%%%%%%%%%%%%%%%%%%%%%%%%%%%%%%%%%%%%%%%%%%%%%%%%%%%%%%%%%%%%%%%%%%%%%%
\section{Inne własności $F_\omega$}
%%%%%%%%%%%%%%%%%%%%%%%%%%%%%%%%%%%%%%%%%%%%%%%%%%%%%%%%%%%%%%%%%%%%%%%%%%%%%%%%
%%%%%%%%%%%%%%%%%%%%%%%%%%%%%%%%%%%%%%%%%%%%%%%%%%%%%%%%%%%%%%%%%%%%%%%%%%%%%%%%
\setcounter{equation}{0}

\subsection{Nierozstrzygalność}

\begin{fakt}
Nierozstrzygalne są problemy:
    \begin{itemize}
        \item sprawdzania typu: dane $\Gamma, M, \tau$, pytamy czy $\Gamma \vdash M : \tau$
        \item typowalność: dane M, pytamy czy $\exists \Gamma \tau . \Gamma \vdash M : \tau$
        \item sprawdzania niepustości typu 
    \end{itemize}
\end{fakt}

\subsection{Redukcja $\mathtt{\Rightarrow}$ na typach - definicja i własności}

Jak już wspomnieliśmy, możemy na typach zdefiniować redukcję, która przebiega bardzo podobnie do $\beta$-redukcji na termach.\\

	\[\mathtt{
		\frac{}{T \Rightarrow T}
                \qquad
                \frac{S_1 \Rightarrow T_1 \;\;\; S_2 \Rightarrow T_2}{S_1 \rightarrow S_2 \Rightarrow T_1 \rightarrow T_2}}
	\]
	\[\mathtt{
                \frac{S_1 \Rightarrow T_1 \;\; S_2 \Rightarrow T_2}{S_1\;S_2 \Rightarrow T_1\;T_2}
                \qquad
                \frac{S \Rightarrow T}{\forall X::K.S \Rightarrow \forall X::K.T} }
	\]
	\[\mathtt{
		\frac{S \Rightarrow T}{\lambda X::K.S \Rightarrow \lambda X::K.T}
                \qquad
                (\lambda X::K.S)T \equiv [X \mapsto T]S}
	\] 
	\[\mathtt{
                \frac{S_1 \Rightarrow T_1 \;\;\; S_2 \Rightarrow T_2}{(\lambda X::K.S_1)S_2 \Rightarrow [X \mapsto T_2]T_1 }}
	\]


\begin{lemat}
Jeśli S$\Rightarrow $T i S $\Rightarrow $U, to istnieje P takie, że T $\Rightarrow$P i U $\Rightarrow$P.
\end{lemat}
\begin{lemat}
Zachodzi własność Churcha-Rossera dla $\Rightarrow$, czyli jeśli S$\Rightarrow ^*$T i S $\Rightarrow ^*$U, to istnieje P takie, że T $\Rightarrow ^*$P i U $\Rightarrow ^*$P.
\end{lemat}
\begin{lemat}
S $\equiv$ T zachodzi wtedy i tylko wtedy, gdy S $\Leftrightarrow^*$T.
\end{lemat}

\subsection{Inne własności}

\begin{twierdzenie}
Jeżeli $\Gamma \vdash t:T$ i $t \rightarrow t'$ to $t':T$.
\end{twierdzenie}

\begin{lemat}
Jeśli $t$ jest zamkniętą wartością i $\vdash t:T_1 \rightarrow T_2$, to $t$ jest abstrakcją.  
\end{lemat}
\begin{lemat}
Jeśli $t$ jest zamkniętą wartością i $\vdash t:\forall X:T_1.T_2$, to $t$ jest abstrakcją typową. 
\end{lemat}

\begin{twierdzenie}
Jeżeli $\vdash t:T$ dla pewnego $T$, to albo $t$ jest wartością, albo jakiś $t'$ taki, że $t \rightarrow t'$.
\end{twierdzenie}

%%%%%%%%%%%%%%%%%%%%%%%%%%%%%%%%%%%%%%%%%%%%%%%%%%%%%%%%%%%%%%%%%%%%%%%%%%%%%%%
%%%%%%%%%%%%%%%%%%%%%%%%%%%%%%%%%%%%%%%%%%%%%%%%%%%%%%%%%%%%%%%%%%%%%%%%%%%%%%%%
%% Bibliografia
%%%%%%%%%%%%%%%%%%%%%%%%%%%%%%%%%%%%%%%%%%%%%%%%%%%%%%%%%%%%%%%%%%%%%%%%%%%%%%%
%%%%%%%%%%%%%%%%%%%%%%%%%%%%%%%%%%%%%%%%%%%%%%%%%%%%%%%%%%%%%%%%%%%%%%%%%%%%%%%%
%\newpage
\thispagestyle{empty}
\begin{thebibliography}{99}

\bibitem{Pie} Pierce B., \textit{Types and programing languages}, The MIT Press, 2002 
\bibitem{Urzy} Urzyczyn P., \textit{Materiały do wykładu Rachunek Lambda} 
\bibitem{MZZ} Mazurak K.,Zhao J.,Zdancevic S., \textit{Lightweight Linear Types in System $F^\circ$}


	  
\end{thebibliography}

\end{document}

